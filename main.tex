% !Mode:: "TeX:UTF-8"

\documentclass[12pt,oneside]{book}

\newlength{\textpt}
\setlength{\textpt}{12pt}
    
\newcommand{\flypage}[1]{\begin{titlepage}\begin{center}\vspace*{\stretch{1}}#1\vspace*{\stretch{1}}\end{center}\end{titlepage}}
    
%========基本必备的宏包========%
\RequirePackage{calc,float,moresize}
%\RequirePackage[onehalfspacing]{setspace}
\linespread{1.5}
%1.3 onehalfspacing
%试卷或需要文字紧凑的
%1.6 doublespacing

%===========加入目录 某章或某节=====%
\makeatletter

\newcommand{\addchtoc}[1]{
        \cleardoublepage
        \phantomsection
        \addcontentsline{toc}{chapter}{#1}}

\newcommand{\addsectoc}[1]{
        \phantomsection
        \addcontentsline{toc}{section}{#1}}

%===========全文基本格式==========%
\setlength{\parskip}{1.6ex plus 0.2ex minus 0.2ex}   %段落間距
\setlength{\parindent}{\textpt * \real{2}}

%=========页面设置=========%
\RequirePackage[a4paper, %a4paper size 297:210 mm
  bindingoffset=10mm,%裝訂線
  top=35mm,  %上邊距 包括頁眉
  bottom=30mm,%下邊距 包括頁腳
  inner=10mm,  %左邊距or inner
  outer=10mm,  %右邊距or  outer
  headheight=10mm,%頁眉
  headsep=15mm,%
  footskip=15mm,%
  marginparsep=10pt, %旁註與正文間距
  marginparwidth=6em,includemp=true% 旁註寬度計入width%旁註寬度
  ]{geometry}

%color
\RequirePackage[table,svgnames]{xcolor}

%================字體================%
%设置数学字体
\RequirePackage{amssymb,amsmath}
\RequirePackage{stmaryrd}
\everymath{\displaystyle}

\RequirePackage{fontspec}
%設置英文字體
\setmainfont[Mapping=tex-text]{DejaVu Serif}
\setsansfont[Mapping=tex-text]{DejaVu Sans}
\setmonofont[Mapping=tex-text]{DejaVu Sans Mono}


%中文環境
\RequirePackage[]{xeCJK}
\xeCJKsetup{PunctStyle=plain}
\setCJKmainfont[FallBack=DejaVu Serif, ItalicFont=TW-Kai]{Source Han Serif CN}
\setCJKsansfont[FallBack=DejaVu Sans]{Source Han Sans CN}
\setCJKmonofont[FallBack=DejaVu Sans Mono]{TW-Kai}


%%===============中文化=========%
\renewcommand\contentsname{目~录}
\renewcommand\listfigurename{插图目录}
\renewcommand\listtablename{表格目录}
\renewcommand\bibname{参~考~文~献}
\renewcommand\indexname{索~引}
\renewcommand\figurename{图}
\renewcommand\tablename{表}
\renewcommand\partname{部分}
\renewcommand\appendixname{附录}
\renewcommand{\today}{\number\year{}年\number\month{}月\number\day{}日}


%=======页眉页脚格式=========%
\RequirePackage{fancyhdr}   %頁眉頁腳
\RequirePackage{zhnumber}  %计数器中文化
\pagestyle{fancy}
\renewcommand{\sectionmark}[1]
{\markright{第\zhnumber{\arabic{section}}节~~#1}{}}

\fancypagestyle{plain}{%
    \fancyhf{}
    \renewcommand{\headrulewidth}{0pt}
    \renewcommand{\footrulewidth}{0pt}
    \fancyhf[HR]{\ttfamily \footnotesize \rightmark }
    \fancyhf[FR]{\thepage}}
\pagestyle{plain}


%=========章節標題設計=========%
\RequirePackage{titlesec}
%修改part
\titleformat{\part}{\huge\sffamily}{}{0em}{}
%修改chapter
\titleformat{\chapter}{\LARGE\sffamily}{}{0em}{}
%修改section
\titleformat{\section}{\Large\sffamily}{}{0em}{}
%修改subsection
\titleformat{\subsection}{\large\sffamily}{}{0em}{}
%修改subsubsection
\titleformat{\subsubsection}{\normalsize\sffamily}{}{0em}{}


%================目录===============%
%toc label to contents space   dynamic adjust
\RequirePackage{tocloft}%
\renewcommand{\numberline}[1]{%
  \@cftbsnum #1\@cftasnum~\@cftasnumb%
}

%==============超鏈接===============%
\RequirePackage[colorlinks=true,linkcolor=blue,citecolor=blue]{hyperref} %設置書簽和目錄鏈接等
\newcommand{\hlabel}[1]{\phantomsection \label{#1}}%某一小段的引用


%=================文字強調=========%
\RequirePackage{xeCJKfntef}

%==================插入圖片=======%
\RequirePackage{wrapfig}
\RequirePackage{graphicx}
\graphicspath{{figures/}}
%change the caption style a little like 1-1
\renewcommand{\thefigure}{\arabic{chapter}-\arabic{figure}}


%==============插入表格========%
\RequirePackage{booktabs}
\renewcommand{\thetable}{\arabic{chapter}-\arabic{table}}
\RequirePackage{caption}
%\renewcommand{\arraystretch}{1.3}
%如果用setspace宏包而不是linespread调整行间距,那么才需要额外的表格行距拉大。

%插入代码
\RequirePackage{fancyvrb} 
\fvset{frame=lines,tabsize=4 ,baselinestretch=1.8, fontsize=\footnotesize}


%==========其他宏包===========%
\RequirePackage{tikz} 
\usetikzlibrary{calc}

%========脚注=========%
\newcommand*\circled[1]{%
  \tikz[baseline=(char.base)]\node[shape=circle,draw,inner sep=0.4pt,minimum size=4pt] (char) {#1};}
\newcommand*\circledarabic[1]{\circled{\arabic{#1}}}

\RequirePackage{perpage} %the perpage package
\MakePerPage{footnote} %the perpage package command

\renewcommand*{\thefootnote}{\protect\circledarabic{footnote}}


\renewcommand\@makefntext[1]
{\vspace{5pt}
\noindent
\makebox[20pt][c]{\@makefnmark}
\fontsize{10pt}{12pt}\selectfont #1}

\setlength{\skip\footins}{20pt plus 10pt}
%main body 与脚注之间的距离


%framed环境
\RequirePackage{framed}

\newenvironment{shici}{
\begin{verse}
\centering\large\hspace{12pt}}
{\end{verse}}

\RequirePackage{indentfirst} 

\makeatother



\title{人生的智慧}
\author{叔本华}
\hypersetup{
  pdfkeywords={},
  pdfsubject={制作者邮箱:a358003542@outlook.com},
  pdfcreator={万泽}}
  
\newcommand{\bookcover}[1]{\tikz[remember picture,overlay]{\node[inner sep=0] at (current page.center)
{\includegraphics[width=\paperwidth,height=\paperheight]{#1}}}} 
 

  
\begin{document}
\frontmatter 

\thispagestyle{empty}

\bookcover{book_cover.png}

\cleardoublepage

\flypage{感谢上天}


\addchtoc{编者前言}
\chapter*{编者前言}
本文是笔者关于叔本华的人生的智慧一书的整理,编辑和点评。具体中文翻译版本是韦启昌翻译的,我对翻译质量不作评论,只是谢谢翻译者并试着从沙砾中发现金矿。按照惯例【】里面的文字为编者所写的文字,脚注里面有很多也是编者所作的工作,然后某些地方编者会自由发挥加上文字着重或者下划线,编者如果精力充沛的话发现某些地方翻译不如人意那么也会试着去找寻原版英文来重新翻译,如此等等不一而足。 

原则上编者不会删减原文,最多是多篇文章组成的文集里面取其精华,同一篇文章一般不会进行原文删减的,最多某些原文中个人认为不是很重要或者不是很认同的文字转为脚注。然后对于原文中某些无关竟要的脚注和译者脚注或者其他无关紧要的附录部分等会做出一些取舍。

原书第五章建议与格言部分有点显得是原出版社凑篇幅而故意加进去的,这里将其舍弃了。


\addchtoc{引言}
\chapter*{引言}
在这本书里,所谓的“人生的智慧”这个术语仅仅只是一般意义上的讨论,其是一门艺术,一门讨论如何尽量幸福、愉快地度过一生的艺术\footnote{形而上者谓之道,形而下者谓之器——周易。原翻译引入什么形而下这样的翻译只是徒增晦涩罢了,这里借鉴英文版的 the common meaning of the term 稍作翻译调整。}。关于这方面的教诲在哲学上可称为“幸福论”。因此,这本著作教导人们如何才能享有幸福的生存。要对“幸福的生存”作出定义的话,那从纯客观的角度考虑,或者更确切地说,通过冷静、缜密的思考(因为这里涉及到主观的判断),这一幸福的生存绝对优于非生存。从这一定义我们就可以这样推论:我们依恋这一生存,就是因为这一生存本身的缘故,而不是出于对死亡的恐惧;并且我们渴望看到这一生存能够永恒地延续。至于人生是否或者能否与如此定义的生存相吻合,这本身就是一个问题。对于这一问题,我的哲学已经清楚无误地给予了否定的答案;但哲学上的幸福论对这一问题却预设了肯定的答案。幸福论的这种肯定答案是基于人的一个与生俱来的错误,这个错误在我的主要著作\footnote{指《作为意欲和表象的世界》。——译者。}的第二卷第四十九章已遭到批判。但要完成诸如幸福论一类的著作,我就只能放弃更高的、属于形而上和道德的审视角度——而我真正的哲学本来就是要引领人们进入这样的审视角度。因此,我在这本书里所作的议论只要是从平常、实用的角度出发,并且保留着与此角度相关的谬误时,那么,这些议论就确实经过了折衷的处理。因此原因,它们的价值就只能是有条件的。其实,Eudamonologie\footnote{亦即幸福论——译者。}这个词本身就是一个委婉词。另外,这些议论还说不上完整彻底——其中的一个原因是我所讨论的主题难以穷尽;另一个原因就是如果我要全面讨论这个主题,那么,我就只能重复别人已经说过的话。【就幸福论这个课题来说本身就包含着一种哲学上的缺乏思辨的错误,这里作者是退而求其次仅仅从实用的角度来做出一些讨论。】 

就我的记忆所及,卡丹奴斯那本颇值一读的《论逆境》其目的与我这本箴言书大同小异。它可以作为我这本书的补充。虽然亚里士多德在他的《修辞学》第一部第五章里,掺进了简短的幸福论方面的论述,但那些只是老生常谈。我并没有并没有这些前辈的著作,因为汇集别人的话语并不是我的工作;况且,如果我这样做了,那我书中的观点就不能一以贯之,而观点的连贯性却是这类著作的灵魂。当然,一般来说,各个时代的智者们,都说过同样的话语,而愚人们——也就是各个时代的数不胜数的大多数人——也做着恰恰相反的同一样事情。因此,伏尔泰说过,“\textbf{当我们离开这个世界的时候,这个世界还是照样愚蠢和邪恶,跟我们刚来到这个世界的时候所发现的并没有两样。}” 



\addchtoc{目录}
\setcounter{tocdepth}{2}    
\tableofcontents



\mainmatter
\chapter{基本的划分}
亚里士多德把人生能够得到的好处分为三类——外在之物、人的灵魂和人的身体。现在我只保留他的三分法。我认为决定凡人命运的根本差别在于三项内容,他们是: 

\begin{description}
\item[人的自身] 即在最广泛意义上属于人的个性的东西。因此,它包括人的健康、力量、外貌、气质、道德品格、精神智力及其潜在发展。 
\item[人所拥有的身外之物] 亦即财产和其他占有物。 
\item[人向其他人所显示的样子] 这可以理解为人在其他人眼中所呈现的样子,亦即人们对他的看法。他人的看法又可分为名誉、地位和名声。 
\end{description}

人与人之间在第一项的差别是大自然确定下来的,由此可推断:这些差别比起第二、三项的差别对于造成人们的幸福抑或不幸福会产生更加根本和彻底的影响——因为后两项内容的差别只是出自人为的划分。人自身拥有的优势,诸如伟大的头脑思想或者伟大的心,与人的地位、出身(甚至王公、贵族的出身)、优厚财富等诸优势相比,就犹如真正的国王比之于戏剧舞台上假扮的国王一样。伊壁鸠鲁的第一个门徒门采多罗斯就曾在他的著作里为他的一个篇章冠以这样的题目:“\textbf{我们幸福的原因存在于我们的自身内在,而不是自身之外。}” 。这是一个明显的事实,都不能称之为一个问题。对于一个人的幸福来说,其一生之中都是如此,最主要的因素就是他由何而造即他的内在\footnote{这里重新翻译了一下:And it is an obvious fact, which cannot be called in question, that the principal element in a man's well-being,—indeed, in the whole tenor of his existence,—is what he is made of, his inner constitution.} 。它直接决定了这个人是否能够得到内心的幸福,因为人的内心快乐抑或内心痛苦首先就是人的感情、意欲和思想的产物。而人自身之外的所有事物,对于人的幸福都只间接发挥影响。因此,同一样外在事物和同一样的境遇对于我们每一个人的影响都不尽相同;就算处在同一样的环境,每一个人都生活在不同的世界中。因为与一个人直接相关的是这一个人对事物的看法、他的感情以及他的意欲活动。外在事物只有在刺激起他的上述东西时才能发挥作用。每个人至於生活于何样的世界首先取决于这个人对这个世界的理解。这个世界因为各人头脑和精神的差异而相应不同。因此,每个人的世界是贫瘠的、浅薄的和肤浅的,抑或丰富多彩、趣味盎然和充满意义——这视各人的头脑而定。例如,不少人羡慕他人在生活中发现和遇到饶有趣味的事情,其实前者应该羡慕后者所具有理解事物的禀赋才对。在后一种人的头脑中,他们所经历过的事情都意味深长,而这一点正可归功于他们的思想禀赋。因为在一个思想丰富的人看来是饶有趣味的事情,对于一个肤浅、庸俗头脑的人来说,同样的事情就只不过是平凡世界里面的乏味一幕而已。这种情形尤其明显见之于歌德和拜伦创作的、明显取材于真实事件的许多诗篇。呆笨的读者会羡慕诗人能有那些其乐无穷的经历,而不是羡慕诗人所具有的伟大的想象力——这种想象力足以化平凡无奇为伟大和优美。同样,一个具忧郁气质的人所看到的悲剧一幕,在一个乐天派的眼里只是一场有趣的冲突,而一个麻木不仁的人则把这视为一件无足轻重的事情。所有这一切都基于这样一个事实:现实生活,亦即当下经历的每时每刻,都由两个部分组成:主体和客体——虽然主体和客体彼此密切关联、缺一不可,就像共同构成水的氧和氢。面对完全一样的客体时,不同的主体就意味着所构成的现实完全不同,反之亦然。由此可知,最美、最好的客体和呆滞、低劣的主体互相结合只能产生出低劣的现实,情形就像恶劣天气之下观赏美丽风景,又或者以糟糕模糊的照相机拍摄这些风景。或者,我们用更浅显的语言来说吧:正如每个人都囿于自己的皮囊,每个人也同样囿于自己的意识。\textbf{一个人只能直接活在自己的意识之中。}因此,外在世界对他帮助不大。在舞台上,演员扮演各种各样的角色:仆人、士兵,或者王侯、将相。但是,这些角色之间的区别只是外在的、皮毛的,这些表面现象之下的内核是一样的;他们都不外乎是可怜、痛苦和烦恼的戏子。在现实生活当中情形也是一样。各人拥有的不同地位和财富赋予了个人不同的角色,但各人的内在幸福并不会因外在角色的不同而产生对应的区别。相反,这些人同样是充满痛苦和烦恼的可怜虫。忧虑和烦恼的具体内容因人而异;但它们的形式,亦即其本质,却大同小异;痛苦和忧虑的程度会存在差别,但这些差别却与人们的地位、财富的差别并不相匹配,亦即和每个人所扮演的角色不相吻合。对于人来说,存在和发生的一切事情总只是直接存在和发生在他的意识里面,所以,很明显,人的意识的构成是首要关键。在大多数情况下,主体意识比呈现在意识中的物象、形态更为重要。一切美妙有趣的事物,经由一个愚人呆滞意识的反映,都会变得枯燥乏味。相比之下,\uline{塞万提斯却在一个简陋牢房里写作了他的《堂吉诃德》}。构成现实的客体部分掌握在命运的手里,因此是可以改变的;但主体部分是我们的自身,所以,就其本质而言它是不可改变的。因此,尽管在人的一生中,外在变化不断发生,但人的性格却始终如一,这好比虽然有一连串的变奏,但主旋律却维持不变。无人能够脱离自身个性。正如那些动物,不管人们把它们放置在何种环境里,它们仍然无法摆脱大自然为它们定下的不可更改的狭窄局限。这一点解释了诸如:为什么我们在努力使自己宠爱的动物快活的时候,应该把这种努力控制在一个狭窄的范围之内,这是由这动物的本性和意识的局限所决定的。人亦如是。一个人所能得到的属于他的快乐,从一开始就已经由这个人的个性规定了。一个人精神能力的范围尤其决定性地限定了他领略高级快乐的能力。如果这个人的精神能力相当有限,那么,所有来自外在的努力——别人或者运气所能为他做的一切——都不会使他超越只能领略平庸无奇、夹杂着动物性的快乐的范围。他只能享受感观的乐趣、低级的社交、庸俗的消费和闲适的家庭生活。甚至教育——如果教育真的有某些用处的话——就大体而言,也无法在拓宽我们精神眼界方面给人带来大的帮助。因为最高级、最丰富多彩以及维持最为恒久的乐趣是精神思想上的乐趣,尽管我们在年轻的时候,对这一点缺乏充足的认识;但是,能否领略这些精神思想的乐趣却首先取决于一个人与生俱来的精神思想能力。从这里可以清楚地看到我们的幸福在多大的程度上取决于我们自身,即取决于我们的个性。但在大多数情况下,我们却只是考虑运气、考虑拥有的财产,或者考虑我们在他人心目中的样子。其实,运气会有变好的时候,甚至如果我们内在丰富的话,我们就不会对运气有太多的要求。相比之下,一个头脑呆滞的人终其一生都是头脑呆滞,一个笨蛋至死仍是一个笨蛋,哪怕他身处天堂,被天堂美女所簇拥着。因此歌德说: 


\begin{quotation}
大众,不分贵贱, 

都总是承认: 

众生能够得到的最大幸运, 

只有自身的个性。 
\end{quotation}


 
对于人的幸福快乐而言,主体远远比客体来得重要,任何一切都可以证实这一点。比如有这些格言:饥饿才是最好的调味品,年少与年长者难以相处,致力于天才和圣贤的生活\footnote{ 这里重新翻译了一下:Hunger is the best sauce, and Youth and Age cannot live together, up to the life of the Genius and the Saint.}。人的健康尤其远远地压倒了一切外在的好处。甚至一个健康的乞丐也的确比一个染病的君王幸运。一副健康、良好的体魄和由此带来的宁静和愉快的脾性,以及活跃、清晰、深刻、能够正确无误地把握事物的理解力,还有温和、节制有度的意欲及由此产生的清白良心——所有这些好处都是财富、地位所不能代替的。一个人的自身,亦即当这个人单独一人的时候陪伴自己的、别人对此不能予夺的内在素质,其重要性明显胜于任何他能够占有的财物和他在他人眼中呈现的样子。一个精神丰富的人在独处的时候,沉浸于自己的精神世界,自得其乐;但对于一个冥顽不灵的人,接连不断地聚会、看戏、出游消遣都无法驱走那折磨人的无聊。一个善良、温和、节制的人在困境中不失其乐;但贪婪、妒忌、卑劣的人尽管坐拥万千财富都难以心满意足。如果一个人能够享有自己卓越的、与众不同的精神个性所带来的乐趣,那么,普通大众所追求的大部分乐趣对于他来说,都是纯属多余的,甚至是一种烦恼和累赘。 

因此,贺拉斯在谈论自己时说: 


\begin{quotation}
象牙、大理石、图画、银盆、雕像、紫衣, 

很多人视它们为必不可少, 

但是有的人却不为这些东西烦心。 
\end{quotation}


\uline{苏格拉底在看到摆卖的奢侈物品时,说道:“我不需要的东西可真不少啊!”}

对我们的生活幸福而言,我们的自身个性才是最重要和最关键的,因为我们的个性持久不变,它在任何情况下都在发挥着作用;另外,它有别于我列出的第二、第三项好处,保存这些好处只能听天由命,但自身个性却不会被剥夺。与后两项只是相对的好处相比较,我们自身的价值,可以说是绝对的。由此可知,通过外在的手段去影响和对付一个人要比人们普遍所认为的困难得多。只有威力无比的时间才可以行使它的权利:人在肉体和精神方面的优势逐渐被时间消磨净尽,也只有人的道德气质不受时间的影响。\sout{在这一方面,财产和别人的看法当然显得更有优势了。因为时间并不会直接夺走这些好处}【时间也会消损财产和改变别人对你的看法的】。它后两项好处的另一个优势就是:因为它们都处于客体的位置,它们的本质决定了任何人都可以得到它们,起码,人们都有占有这些好处的可能。相比之下,对于属于主体的东西我们确实无能为力——它们是作为“神的权利”赋予了人们,并终生牢固不变。所以歌德说: 

 
\begin{quotation}
在你降临世上的那一天; 

太阳接受了行星的问候, 

你随即永恒地遵循着, 

让你出世的法则茁壮成长, 

你就是你,你无法逃脱你自己, 

师贝尔和先知已经这样说过; 

 时间,力量都不能打碎, 

那既成的、已成活的形体。 
\end{quotation}


\uline{我们唯一能够做到的就是尽可能充分地利用我们既定的个性。因此,我们应该循着符合我们个性的方向,努力争取适合个性的发展,除此之外则一概避免。所以,我们必须选择与我们个性相配的地位、职业和生活方式。}

一个天生筋骨强壮,长得像大力神似的人,如果为外在情势所迫,需要从事某种坐着的职业,去做一些精细、烦琐的手艺活,或者从事学习研究和其他脑力工作——这些工作需要他运用先天不足的能力,而他那出色的身体力量却无从发挥——要是出现这种情况,那这个人终其一生都会感到郁郁不得志。但如果一个人虽然具有异常突出的智力,但其智力却无从得到锻炼和发挥,从事的是一种根本发挥不了他的智力的平庸工作;或者,这工作干脆就是他力不能及的力气活,那么,这个人遭遇的不幸比起第一个人则有过之而无不及。所以我们必须避免过高估计自己的能力,尤其在我们年少气盛的时候,这可是我们生活中的暗礁。 【我们常常忙于生活而忘了自己。认识你自己!】

人的自身比起财产和他人对自己的看法具有压倒性的优势;由此可知,\uline{注重保持身体健康和发挥个人自身才能比全力投入获得财富更为明智}。但我们不应该把这一说法错误地理解为:我们应该忽略获得我们的生活必需品。不过真正称为财富的,亦即过分的丰裕盈余,对我们的幸福却帮助不大。所以,很多的有钱人感觉并不快乐,因为这些不快乐的有钱人缺乏真正的精神思想的熏陶,没有见识,也因此缺乏对事物的客观兴趣——而只有这些才可以使他们具备能力从事精神思想的活动。财富除了能满足人的真正、自然的需求以外,对于我们的真正幸福没有多大影响。相反,为了保管好偌大的财产,我们会有许多不可避免的操劳,它们打扰了我们舒适悠闲的生活。对于人的幸福,人的自身确实较之于人所拥有的财富更为重要,但是,常人追求财富比追求精神情趣要来劲千百倍。因此我们看到很多人像蚂蚁似地不眠不休、辛勤劳作,从早到晚盘算着如何增加他们已有的财富。一旦脱离了那狭窄的挣钱领域,他们就一无所知。他们的精神空白一片,因此对挣钱以外的一切事物毫无感知。人生最高的乐趣——精神方面的乐趣——对他们来说,是遥不可及的事情。既然如此,他们就只能忙里偷闲地寻求那些短暂的、感官的乐趣——它们费时很少,却耗钱很多。他们徒劳地以这类娱乐来取代精神上的享受。在他们生命终结的时候,如果运气好的话,他们真的会挣到一大堆的金钱,这是他们一生的成果;他们就会把这钱留给自己的继承人去继续积累或者任意挥霍。这种人尽管终其一生都板着一副严肃、煞有介事的面孔,但他们的生活仍然是愚不可及的,与其他许多傻乎乎的人生没有多少两样。 

所以,人的内在拥有对于人的幸福才是最关键的。正因为在大多数情形下人的自身内在相当贫乏,所以,那些再也用不着与生活的匮乏作斗争的人,他们之中的大多数从根本上还是感觉到闷闷不乐。情形就跟那些还在生活的困苦中搏斗的人一般无异。他们内在空虚、感觉意识呆滞、思想贫乏,这些就驱使他们投入到社交人群中。组成那些社交圈子的人也正是他们这一类的人,“因为相同羽毛的鸟聚在一块”(荷马语)。他们聚在一块追逐消遣、娱乐。他们以放纵感官的欢娱、极尽声色的享受开始,以荒唐、无度而告终。众多刚刚踏入生活的纨绔子弟穷奢极欲,在令人难以置信的极短时间内就把大部分家财挥霍殆尽。这种作派,其根源确实不是别的,正是无聊——它源自上述的精神贫乏和空虚。一个外在富有、但内在贫乏的富家子弟来到这个世界,会徒劳地试图用外在的财富去补偿内在的不足;他渴望从外部得到一切,这情形就好比一个老朽之人寻求健壮的体魄如同大卫王或Maréchal de Rex试着去做的那样\footnote{这里重新翻译了一下,英文为:like an old man who seeks to strengthen himself as King David or Maréchal de Rex tried to do. 因为原作品这里并没有注解,我不确定这里谈及的是大卫王的那个典故,但还是将尊重原文的翻译放置在此处,而不强行杜撰。}。人自身内在的贫乏由此导致了外在财富的贫乏。 

至于另外两项人生好处的重要性,不需要我特别强调。财产的价值在当今是人所公认的,用不着为其宣传介绍。比起第二项的好处,第三项的好处具有一种相当缥渺的成分,因为名誉、名望、地位等全由他人的意见构成。每人都可以争取得到名誉,亦即清白的名声;但社会地位,则只有那些国家政府的人才能染指;至于显赫的名望就只有极少数人才会得到。在所有这些当中,名誉是弥足珍贵的;显赫的名望则是人所希望得到的价值至昂的东西,那是天之骄子【底层谋食,中层谋名,高层谋权。所谓天之骄子所谓精英,吾视之若无物。】才能得到的金羊毛。另一方面,只有傻瓜才会把社会地位放置在财产之前。另外,人拥有的财产、物品和名誉、声望是处于—种所谓的互为影响、促进的关系。彼德尼斯说过:“\uline{一个人所拥有的财产决定了这个人在他人眼中的价值。}”如果这句话是正确的话,那么,反过来,他人对自己的良好评价,能以各种形式帮助自己获取财产。 

  

\chapter{人的自身} 
一个人的自身比起这个人所拥有的财产或者他所给予别人的表象都更能带给他幸福——这一点我们已经大致上认识到了。一个人本身到底是什么,也就是说,他自身所具备的东西,才是最关键的,因为一个人的自身个性永远伴随着他,他所体验的一切都沾上他的个性的色彩。无论他经历何种事情,他首要感受到的是他自己。这一点适用于人们从物质事物中获取的乐趣,而享受精神上的乐趣则更是如此。因此,英语的短语to enjoy one's self(享受)是一个相当生动的表述。例如:人们说:“He enjoys himself in Paris”(他在巴黎享受自己),而不是说“他享受巴黎”。如果一个人的自身个性相当低劣,那么所有的乐趣都会变味,就像把价值不菲的美酒倒进被胆汁弄得苦涩难受的嘴里一样。因此,除了严重灾祸以外,人们在生活中所遭遇到的事情,不论是好是坏,其重要性远远不及人们对这些事情的感受方式;也就是说,人们对事情的感受能力的本质特性和强弱程度才更为重要。一个人的自身是什么,他的自身拥有到底为何,简而言之:他的个性及其价值才唯一直接与他的幸福有关。除此之外的一切都只是间接发挥作用,这些作用因此是可以消除的。但个性发挥的作用却永远无法消除。

因此,针对他人自身优点而产生的嫉妒是最难消除的;所以这种嫉妒会被很小心、谨慎地掩藏起来。进一步而言,只有感觉意识的构成才是恒久保持的,人的个性每时每刻都持续地发挥着作用;相比较而言,除此以外的任何其他东西都永远只是暂时地、偶尔地产生作用,并且它们都受制于不断发生的各种变化。所以,亚里士多德说过:“我们能够依靠的只是我们的本性,而不是金钱。”正因为这样,我们能够咬紧牙关承受纯粹从外而至的灾祸,但由我们的自身所招致的不幸却更难忍受;\sout{因为运气会有变好的时候,但我们的自身构成却永远不会改变}【自身构成也是在变化之中的。】。因此,对于人的幸福起着首要关键作用的,是属于人的主体的美好素质,这些包括高贵的品格、良好的智力、愉快的性情和健康良好的体魄——一句话,“\textbf{健康的身体加上健康的心灵}”(尤维纳利斯语)。所以我们应该多加注意保持和改善这一类的好处,而不是一门心思只想着占有那些身外的财产、荣誉。 

在上述这些主体的美好素质当中,最直接带给我们幸福的莫过于轻松、愉快的感官。因为这一美好的素质所带来的好处是即时呈现的,一个愉快的人总有他高兴愉快的原因,原因就是:他就是一个愉快的人。一个人的这种愉快气质能够取代一切别的内在素质,但任何别的其他好处都不可以替代它。一个人或许年轻、英俊、富有和备受人们的尊重,但如果要判断这个人是否幸福,那我们就必须问一问自己:这个人是否轻松愉快?如果他心情愉快,那么,他是年轻抑或年老,腰板挺直抑或腰弯背驼,家财万贯抑或一贫如洗——这些对他而言,都是无关重要的:反正他就是幸福的。我在年轻的时候,有一次翻开了一本旧书,赫然入目的是这样一句话:“\textbf{谁经常笑,谁就是幸福的;谁经常哭,谁就是痛苦不幸的。}”这是一句再普通不过的话了,但我却一直无法把它忘记,因为这句话包含着朴素的真理,虽然这老生常谈说得夸张了点。\uline{因此,当愉快心情到来之时,我们应该敞开大门欢迎它的到来,因为它的到来永远不会不合时宜。}但我们往往不是这样做:我们经常会犹豫不决地接受愉快的心情——我们想先弄清楚我们的高兴和满足是否确有根据。又或者,我们担心在严肃地盘算和认真地操劳之际,高兴的心情会打扰了我们。其实,这种做法是否真有好处仍是一个未知数。\uline{相比之下,高兴的心情直接就使我们获益。它才是幸福的现金,而其他别的都只是兑现幸福的支票。}高兴的心情在人们感受高兴的此时此刻就直接给人以愉快。所以,对于我们的生存,它是一种无与伦比的恩物,因为我们生存的真实性就体现在此时此刻——它无法割裂地连接无尽的过去和将来。由此可见,我们应把获得和促进愉快的心情放在各种追求的首位。确实,能够增进愉快心情的莫过于健康;而对于愉快心情贡献最小的则是充裕盈余的金钱财富。那些低下的劳作阶层,特别是在乡下生活的人们,常常露出高兴和满足的表情,而富贵人家却通常感到烦恼。因此,\textbf{我们应该着重获得和保持身体健康}——愉快的心情就是从健康的身体里长出的花朵。众所周知,保持身体健康的手段无非就是避免一切纵欲放荡的行为、令人不快和剧烈的情绪动荡,以及长时间、紧张的精神劳累;每天至少在户外进行两个小时的身体快速运动;勤洗冷水浴,饮食有节。如果一个人每天不进行一定的身体活动,那他就无法保持健康。一切生命活动程序,如果要保持运作正常的话,那么,生命活动程序所在的整体也好,作为这一整体里面的一部分也好,都需要得到运动。因此,亚里士多德说得很对,“\textbf{生命在于运动,生命的本质在于运动。}” 身体组织的内部在永不停歇地快速运动;心脏在复杂的双重收缩和舒张的过程中,强劲地、不知疲倦地跳动;心脏每跳动28次,就把身体的全部血液沿着身体的大、小血脉传送一遍,肺部一刻不停地抽气,就像一台蒸汽机;大肠则像虫子一样地蠕动不已;体腺始终在吸收和排泄;伴随着一次脉搏跳动和每一次呼吸,大脑本身就完成了一次双重运动。这样,如果人不进行外在的运动——很多人的生活方式都是静止缺少运动的——那他们身体外表的静止就会与内在进行着的运动形成惊人的、有害的不协调。身体内部不停的运动需要得到某种外在运动的配合与支持。上述身体内外之间的不协调就类似于:某种情绪使我们的内在沸腾激动起来,但却不得不竭力压制这种情绪从我们外表流露出来。甚至树木的生长茂盛也必须借助风的吹动。“每一运动的速度越快,那这一运动就越成其为运动”——这一句话以最简洁的拉丁文表示,就是“Omnis motus guo celerior,eo magis motus”——这一规则可以适用在这里。我们的幸福取决于我们的愉快情绪,而愉快情绪又取决于我们身体的健康状况。关于这点,只要互相对照一下我们在健康、强壮的日子里和当疾病降临、我们被弄得苦恼焦虑的时候,外在境况和事件所留给我们的不同的感觉印象,一切就都清楚了。使我们快乐或者忧伤的事物,不是那些客观、真实的事物,而是我们对这些事物的理解和把握。这就是爱比克泰德所说的“扰乱人们的不是客观事情,而是人们对客观事情的见解”。


我们的幸福十占其九依赖于我们的健康。只要我们保持健康,一切也就成了快乐的源泉;但缺少了健康,一切外在的好处——无论这些好处是什么——都不再具有意义,甚至那些属于人的主体的好处,诸如精神思想、情绪、气质方面的优点等,仍会由于疾病的缘故而被大打折扣。由此看来,人们在彼此相见时首要询问对方的健康状况,并祝愿对方身体健康的做法也就不是没有根据的了,因为健康对于一个人的幸福的确是头等重要的事情。我们可以由此得出这样的结论:\textbf{最大的愚蠢也就是为了诸如金钱、晋职、学问、声名,甚至为了肉欲和其他片刻的欢娱而献出自己的健康。我们更应该把健康放在第一位。}

虽然健康能极大地增进我们的愉快心情——这种愉快心情对于我们的幸福头等重要——但愉快的心情却不完全依赖于健康;因为即使是完全健康的人也会生成忧郁的气质和沮丧的心情。在这里,最根本的原因无疑在于人的最原初的、因而也是不可改变的机体组织的构成;也就是说,大致上在于一个人的感觉能力与肌肉活动、兴奋能力及机体新陈代谢能力之间构成的正常程度不一的比例。超常的感觉能力会引致情绪失衡、周期性的超乎寻常的愉快或者挥之不去的忧郁。天才的条件就是具备超越常人的神经力量——亦即超常的感觉能力。所以,亚里士多德相当正确地认为:所有杰出、优越的人都是忧郁的:“所有那些无论是哲学、政治学、诗歌或其他艺术方面表现出色的人,看上去都是忧郁的”。西塞罗在讲述下面这句经常被人们引用的话时,他所指的肯定也是上述那段话:“亚里士多德说,所有的天才人物都是忧郁的。”我在这里对人的与生俱来的基本情绪——它因人而异——所作的考察,莎士比亚曾经异常优美地加以描述: 

 
\begin{quotation}
大自然造就了奇特的人,

一些人总是眯缝着眼睛,大声笑着, 

就像看见苏格兰风笛手的鹦鹉; 

也有一些人阴沉着面孔,笑不露齿, 

虽然奈斯特发誓那笑话的确值得一笑。 

——《威尼斯商人》 
\end{quotation}


柏拉图用了“郁闷”和“愉快”这样的词语来形容这两种不同情绪,出现这些不同情绪是因为不同的人有着极为不同的感受愉快和不愉快印象的能力。因此,一件使一个人近乎绝望的事情,会让另一个人高兴发笑。一般而言,一个人接受愉快印象的能力越弱,那他接受不愉快印象的能力也就越强,反之亦然。同一件事情有出现好或不好两种结果的可能。“郁闷”型的人会因为“不好”的结果而感到悲哀和烦躁,对好的结果也提不起高兴劲儿。“愉快”型的人却不会为不幸的结果悲哀和烦恼,但对事物的好结果却会深感高兴。对“郁闷”型的人来说,尽管他们实现了十个目标中的九个,他们仍然不会为已实现了的目标高兴,而仅仅因为一个目标的落空而烦恼、生气。愉快型的人则相反,他们会从成功实现了的目标那里取得安慰和愉快。【愉快型的人想到明天还有面包吃就知足了,郁闷型的人明年甚至几十年之后有没有面包吃也是他们忧愁伤心的原因。】不过,正如没有一丁点好处的十足坏事并不容易找到,同样,“郁闷”型的人,亦即阴沉和神经兮兮的人,虽然总的来说比无忧无虑、快乐的人承受更多只是想象出来的不幸和苦难,但却因此而遭遇更少真实的不幸和苦难因为他们把一切都看成漆黑一团,总是把事情往最坏的方面想,并因此准备着防范措施。这样,与那些总是赋予事情以愉快色彩和大好前景的人相比,他们更少失算与栽跟斗。但如果一个天生具有不满、易怒心态的人,再加上神经系统或者消化器官疾病的折磨,情况最终可以发展成这个样子:持续的不幸引致了对生活的厌烦,并由此萌生了自杀的倾向。由于这个原因,最微不足道的不便和烦恼都会引致自杀的结果。的确,当情况变得最糟糕的时候,甚至连这点不便和烦恼也不需要了,一个人会纯粹由于持续闷闷不乐的心情而决定自杀。这种人会以冷静的思考和铁定的决心实施自杀行为。这种情况经常发生:一个病人尽管处于别人的监视之下,仍会随时留意着利用每个不被监视的机会,迫不及待地抓住这现在对于他来说是求之不得的和最自然不过的解脱痛苦的手段——整个过程没有犹豫、退缩和内心斗争。关于自杀方面的详尽论述,可阅读埃斯基罗尔的《精神疾病》一书。但除此之外,在某种情况下,就算是最健康的和或许是最愉快的人也会想到过自杀。那就是当痛苦非常巨大,或者,步步逼近的不幸实在不可避免,这一巨大的痛苦或不幸已经压倒了对死亡的恐惧。不同之处只在于自杀所必需的诱因的大小,这一诱因和人的不满情绪成反比例。不满情绪越厉害,则自杀所需的诱因就越小,到最后,诱因可以减至为零。相比之下,愉快情绪越强烈,维持这一情绪的健康状况越良好,自杀的诱因就必须越大。因此,导致自杀的原因大小不一,但构成两个极端的就是:与生俱来的忧郁不满的心理得到了病态的加剧;天性是健康、愉快的,只是客观的原因所致。 

健康和美貌有着部分的关联,虽然美貌这一属于主体的好处不会直接带给我们幸福——它只是间接通过留给别人印象的方式做到这点——但美貌仍然是至为重要的,甚至对男人来说也是如此。良好的长相是一纸摊开的推荐书,它从一开始就为我们赢得了他人的心。因此,荷马这些诗句尤其适用于我在这里所说的话: 

 
\begin{quotation}
神祇的神圣馈赠不容遭到蔑视, 

这些馈赠只能经由神祇的赐予。 

任何人都无法随心所欲地获取它们。 

——《伊利亚特》 

\end{quotation}

 

对生活稍作考察就可以知道:\uline{痛苦和无聊是人类幸福的两个死敌},关于这一点,我可以作一个补充:每当我们感到快活,在我们远离上述的一个敌人的时候,我们也就接近了另一个敌人,反之亦然。所以说,我们的生活确实就是在这两者当中或强或弱地摇摆。这是因为痛苦与无聊之间的关系是双重的对立关系。一重是外在的,属于客体;另一重则是内在的,属于主体。外在的一重对立关系其实也就是生活的艰辛和匮乏产生出了痛苦,而丰裕和安定就产生无聊。因此,我们看见低下的劳动阶层与匮乏——亦即痛苦——进行着永恒的斗争,而有钱的上流社会却旷日持久地与无聊进行一场堪称绝望的搏斗。而内在的或者说属于主体的痛苦与无聊之间的对立关系则基于以下这一事实:一个人对痛苦的感受能力和对无聊的感受能力成反比,这是由一个人的精神能力的大小所决定的。也就是说,一个人精神的迟钝一般是和感觉的迟钝和缺乏兴奋密切相关的,因此原因,精神迟钝的人也就较少感受到各种强度不一的痛苦和要求。但是,精神迟钝的后果就是内在的空虚。这种空虚烙在了无数人的脸上。并且,人们对于外在世界发生的各种事情——甚至最微不足道的事情——所表现出的一刻不停的、强烈的关注,也暴露出他们的这种内在空虚。人的内在空虚就是无聊的真正根源,它无时无刻不在寻求外在刺激,试图借助某事某物使他们的精神和情绪活动起来。他们做出的选择真可谓饥不择食,要找到这方面的证明只须看一看他们紧抓不放的贫乏、单调的消遣,还有同一样性质的社交谈话,以及许许多多的靠门站着的和从窗口往外张望的人。正是由于内在的空虚,他们才追求五花八门的社交、娱乐和奢侈;而这些东西把许多人引人穷奢极欲,然后以痛苦告终。使我们免于这种痛苦的手段莫过于拥有丰富的内在——即丰富的精神思想。因为\uline{人的精神思想财富越优越和显著,那么留给无聊的空间就越小}。这些人头脑里面的思想活泼奔涌不息,不断更新;\uline{它们玩味和摸索着内在世界和外部世界的多种现象;还有把这些思想进行各种组合的冲动和能力}——所有这些,除了精神松弛下来的个别时候,都使卓越的头脑免受无聊的袭击。但是,突出的智力是以敏锐的感觉为直接前提,以强烈的意欲,亦即强烈的冲动和激情为根基。这些素质结合在一起提高了情感的强烈程度,造成了对精神,甚至肉体痛苦的极度敏感。对任何不如意的事情,甚至细微的骚扰,都会感觉极度不耐烦。所有这些素质大大加强了头脑里面事物的各种表象,包括拂逆人意的东西。这些表象由于头脑强烈的想象力的作用而变得生动活泼。我这里所说的比较适用于所有各种精神思想能力参差不一的人,从最呆笨的头脑一直到最伟大的思想天才。由此可知,无论从客体抑或从主体上说,如果一个人距离人生痛苦的其中一端越近,那他距离痛苦的另一端也就越远。据此,每个人的天性都会指导自己尽可能地调节客体以适应主体,因而更充足地做好准备以避免自己更加敏感的痛苦一端。一个精神富有的人会首先寻求没有痛苦、没有烦恼的状态,追求宁静和闲暇,亦即争取得到一种安静、简朴和尽量不受骚扰的生活。因此,一旦对所谓的人有所了解,他就会选择避世隐居的生活;如果他具备深邃、远大的思想,他甚至会选择独处。因为一个人自身拥有越丰富,他对身外之物的需求也就越少,别人对他来说就越不重要。所以,一个人具备了卓越的精神思想就会造成他不喜与人交往。的确,如果社会交往的数量能够代替质量,那么,生活在一个熙熙攘攘的世界也就颇为值得的了。但遗憾的是,\uline{一百个傻瓜聚在一起,也仍然产生不了一个聪明的人}。相比之下,处于痛苦的另一极端的人,一旦匮乏和需求对他的控制稍微放松,给他以喘息的机会,他就会不惜代价地寻找消遣和人群,轻易地将就一切麻烦。他这样做的目的不为别的,只是为了逃避他自己。因为在独处的时候,每个人都只能返求于自身,这个人的自身具备就会暴露无遗。因此,一个愚人背负着自己可怜的自身——这一无法摆脱的负担——而叹息呻吟。而有着优越精神思想禀赋的人却以其思想使所处的死气沉沉的环境变得活泼和富有生气。因此,塞尼加所说的话是千真万确的:“愚蠢的人受着厌倦的折磨”。同样,耶稣说:“愚人的生活比死亡还要糟糕。”因此,我们可以发现:大致而言,一个人对与人交往的爱好程度,跟他的智力的平庸及思想的贫乏成正比。人们在这个世界上要么选择独处,要么选择庸俗,除此以外再没有更多别的选择了。【独处中的人也不一定会返求自身,有很多消遣可以让现代人打发时间迷失自我。这里谈论的重点是\textbf{对自我的内在精神世界的追寻}。只有丰富了自我的内在精神世界才能实际地免于无聊和空虚。】

人的大脑意识是人的身体的寄生物,它寓寄在人的身体之中,而人们辛苦挣来的闲暇,就是为了让人能够自由地享受意识和个性所带来的乐趣。所以,\textbf{闲暇是人生的精华},除此之外,人的整个一生就只是辛苦和劳作而已。但闲暇给大多数人带来了什么呢?如果不是声色享受和胡闹,就是无聊和浑噩。人们消磨闲暇的方式就显示出闲暇对于他们是何等的没有价值。他们的闲暇也就是阿里奥斯托所说的“一无所知者的无聊”。凡夫俗子只关心如何去打发时间,而略具才华的人却考虑如何好好利用时间。头脑思想狭隘的人容易受到无聊的侵袭,其原因就是他们的智力纯粹服务于他们的意欲,是意欲的工具。如果诱发意欲的动因暂时没有出现,那么,意欲就休息了,智力也就放假了——因为智力和意欲不一样,它不会自动活动起来。这样,人身上的所有力量可怕地迂滞静止,这也就是无聊。为了应付无聊,人们就为意欲找出一些琐碎、微小、随意和暂时的动因以图刺激意欲,和以此激活智力——因为智力的任务本来就是理解、把握动因。但这些动因较之于那些真正的、自然的动因,就犹如纸币比之于银元,因为前者的价值是有随意性的;诸如游戏、玩纸牌等就属于这一类的动因。这些游戏的发明也就是为了上述目的。如果没有了这些游戏,缺乏思想的人就会敲击随便一件伸手可及的物品来帮助自己打发时光。对这种人而言,雪茄同样是一件受欢迎的代替思考的物品。因此,在各国,玩纸牌成了社交、聚会的主要娱乐。它反映了这种社交聚会的价值,也宣告了思想的破产。因为人们彼此之间没有可以交换的思想,所以,他们就交换纸牌,并试图赢取对方的金钱。可怜的人啊!但我不想有欠公正地压制这样的想法,那就是我们可以为玩纸牌游戏作这样的辩护:玩纸牌不失为一种应付以后的世俗生活的演习,只要我们通过玩牌能学习到如何巧妙地运用那听任偶然的、不可更改的既定形势(牌局),使我们尽量得到我们所能得到的东西;为此目的,人们必须养成习惯保持沉着,即使牌势恶劣的时候,仍能装出一副高兴的外表。不过,正因为这样,玩牌也就会产生一种伤风败俗的作用。这种游戏的特质就在于人们动用一切诡计和技巧,不择手段地去赢取他人的财物。这种在游戏里面体验和获得的习惯,会在人的实际生活里生根、蔓延。这样,人们逐渐在处理人与人之间的事务中,也同样依照这种习惯行事,认为只要法律允许,就可以利用掌握在手的每一个优势。这方面的例证在日常生活中俯拾皆是。正如我已经说过的,闲暇就是每一个人的生命存在开出的花朵,或者毋宁说是果实。也只有闲暇使人得以把握、支配自身,而那些自身具备某些价值的人才可以称得上是幸福的。但对于大多数人来说,闲暇只会造就一个无用的家伙,无所事事,无聊烦闷,他的自身变成了他的包袱。因此,我们应该庆幸:“亲爱的兄弟们,我们不是干粗活女工的孩子,我们是自由的人”。 

进一步而言,正如一个不需要或只需要很少进口物品的国家是最幸运的国家。同样,如果一个人内在充足、丰富,不需要从自身之外寻求娱乐,那么,他就是一个最幸运的人。因为进口物品使国家花费不菲,仰仗他人,同时又带来危险、制造麻烦。到头来,这些物品只能是我们本土产品的糟糕的代替品,因为无论如何,<u>我们不应该从他人那里,或者从自身之外期望太多。</u>一个人对另一个人所能做的只是极为有限。归根到底,每个人都孑然独立,这时候,最关键的问题就是这单独的人是个什么样的人。因此,歌德的评论(《诗与真》)适用于这里:无论经历任何事情,每个人都最终都得返求于己。或者,就像奥立弗·高尔斯密的诗句说的: 

 
\begin{quotation}
无论身在何处, 

我们只能在我们自身寻找或者获得幸福 

——《旅行者》 
\end{quotation}


因此,每个人都要充分发挥自己的所能,努力做到最好。一个人越能够做到这一点,那他在自己的身上就越能够发现快乐的源泉,那他也就越幸福。亚里士多德无比正确地说过:**幸福属于那些能够自得其乐的人。**这是因为幸福和快乐的外在源泉,就其本质而言,都极其不确定,并且为时短暂和受制于偶然。因此,甚至在形势大好的情况下,它们仍然会轻易终结。的确,只要这些外在源泉不在我们的控制之下,那这种情形就是不可避免的。人到老年,几乎所有这些外在源泉都必然地干枯了,因为谈情说爱、戏谑玩笑、对旅行的兴趣、对马匹的喜好,以及应付社交的精力都舍我们而去了;甚至我们的朋友和亲人也被死亡从我们的身边一一带走。此时此刻,一个人的自身拥有,比起以往任何时候都更加重要,因为我们的自身拥有能够保持得至为长久。不过,无论在任何年龄阶段,一个人的自身拥有都是真正的和唯一持久的幸福源泉。我们这个世界乏善可陈,到处充斥着匮乏和痛苦,对于那些侥幸逃过匮乏和痛苦的人们来说,无聊却正在每个角落等待着他们。此外,在这个世界上,卑劣和恶毒普遍占据着统治的地位,而愚蠢的嗓门叫喊得至为响亮,他们的话语也更有分量。命运是残酷的,人类又是可怜可叹的。生活在这样的一个世界里,一个拥有丰富内在的人,就像在冬月的晚上,在漫天冰雪当中拥有一间明亮、
温暖、愉快的圣诞小屋。因此,能够拥有优越、丰富的个性,尤其是深邃的精神思想,无疑就是在这地球上得到的最大幸运,尽管命运的发展结果不一定至为辉煌灿烂。因此,年仅19岁的瑞典克里斯汀女王在评论笛卡尔时——她只是通过笛卡尔的一篇论文以及一些口头资料了解到这位已经在荷兰孤独生活了20年的人——说了一句充满睿智的话:笛卡尔先生是我们所有人当中最幸福的一个;在我看来,他的生活令人羡慕(《笛卡尔的一生》,巴叶著)。当然,就像笛卡尔的情形那样,外部环境必须允许我们支配自身,并从中汲取快乐。所以圣经《传道书》已经说过:“智慧再加上一笔遗产就美好了,智慧帮助一个人享受阳光。”谁要是通过大自然和命运的恩赐,交上好运得到内在的财富,那他就要小心谨慎地确保自己幸福的这内在源泉畅通无阻。但要达到这一目的,条件就是拥有独立和闲暇。因此,这种人会乐意以俭朴和节制换取上述二者。如果他们不像其他人那样依赖快乐的外在源泉,情况就更是如此。因此,对职位、金钱、世人的赞许和垂青等诸如此类的指望终究不会把这种人诱入歧途,牺牲自己以迎合人们卑微的目的或者低下的趣味。有机会的话,他就会像贺拉斯在给默斯那斯的信中所建议的那样做。为了外在的荣耀、地位、头衔和名声而部分或全部地奉献出自己的内在安宁、闲暇和独立——这是极度的愚蠢行为。歌德就是这样做了。但我的守护神却明确地指引我走向与此相反的方向。 

我们在这里讨论的真理,即幸福源自于人的内在,被亚里士多德的真知灼见所引证(《伦理学》)。他说:每一快乐都是以人从事某种活动,或者应用人的某种能力作为前提;没有这一前提,快乐也就无从谈起,亚里士多德的教导——即人的幸福全在于无拘束地施展人的突出才能——与斯托拜阿斯对逍遥派伦理学的描述如出一辙。斯托拜阿斯说:“幸福就是发挥、应用我们的技巧,并取得期待的效果。”他特别说明他所用的古希腊字词指的是每一种需要运用技巧和造诣的活动。大自然赋予人们以力量,其原始目的就是使人能够和包围着人们的匮乏作斗争。一旦这场斗争停止了,那再也派不上用场的力量就会成为人的负担。因此,他必须为它们找到消遣,亦即不带任何目的地运用这些力量。因为如果不这样做,人就会马上陷入人生的另一个痛苦——无聊——之中。因此,王公、巨富尤其受到无聊的折磨。关于他们的痛苦,卢克莱修留给我们这样一段描写。当今我们在每个大城市,每天都有机会见到类似的例子: 

 
\begin{quotation}
他经常离开偌大的宫殿,匆匆走向室外露天——因为在屋子里他感到厌烦——直到他突然返回为止,因为他感觉出门并没有好得了多少。又或者,他策马驰往乡村庄园,就好像他的庄园燃起了大火,他必须匆忙赶去扑救一样。但刚跨进乡村庄园的门槛,他就无聊地呵欠连连,或者干脆倒头大睡。他要尽力去忘记自己,直到他想返回城市为止。 
\end{quotation}
 

这些先生们在年轻的时候,肌肉力量和生殖能力都旺盛十足。但随着岁月的流逝,只有思想能力才能保留下来。如果我们的思想能力本身就有所欠缺,或者,我们的思想能力没有得到应有的锻炼,又或者,我们欠缺能发挥思想能力的素材,那我们将遭遇到的悲惨情形就着实令人同情。意欲是唯一无法枯竭的力量,它受到激情的刺激就会抬头。例如,意欲可以通过一掷千金的豪赌——这一真正低级趣味的罪恶——而被鼓动起来。一般来说,每个无事可做的人都会挑选一种能够运用自己的特长的消遣,比如下棋、玩牌、狩猎、绘画、赛马、玩九柱戏;或者研究文章、音乐、诗歌或者哲学。我们可以探索人的能力的所有外在表现的根源,亦即深入到人的三种生理基本能力,从而对这个课题有一个彻底的了解。我们也就需要考察这三种能力的那些不带目的的发挥和活动——它们的发挥和活动构成了人的三类快乐的源泉。每个人都会有适合自己的一类快乐,这由他身上所突出具备的是哪一种能力而定。第一类是为机体新陈代谢能力所带来的乐趣:这包括吃喝、消化、休息和睡觉。在一些国家,这类快乐获得首肯,这类活动甚至成为全民性的娱乐。第二类是发挥肌肉力量所带来的乐趣:这些包括步行、跳跃、击剑、骑马、舞蹈、狩猎和各种各样的体育游戏;甚至打斗和战争也包括在内。第三类为施展感觉能力方面的乐趣:这些包括观察、思考、感觉、阅读、默想、写作、学习、发明、演奏音乐和思考哲学等。关于这各种各样的乐趣的等级和价值,以及它们所能维持的时间,会有诸多说法,读者们也尽可以作出补充。但我们应该清楚:我们感受的乐趣(它以运用、发挥我们的能力为前提)和幸福(它由乐趣的不断重复所构成)越大,那作为前提的力量也就越高级。并且,没有人会否认,在这一方面,感觉能力比人的另外两种基本生理力量更为优越——人较之于动物在感觉方面的明显优势就是人优胜于动物之处,但人的另外两种基本生理能力在动物身上也同样存在,甚至远胜于人类。感觉能力隶属于人的认知能力;因此,卓越的感觉力使我们有能力享受到属于认知的,亦即所谓精神思想上的乐趣。情感能力越卓越和明显,那么,我们所享受到这方面的乐趣就越大\footnote{大自然持续不断地演变。大自然首先是无机王国的机械和化学活动,接着是植物王国,以及植物的那些麻木的自我陶醉;再接下来就发展到了动物王国。在动物的身上,智力和意识朦胧初开。大自然的发展是从低级开始,逐步迈向更高的一级。到最后,她终于迈出了最终的和最伟大的一步,从而达到了人的级别。人所具备的智力就是大自然发展到了登峰造极阶段的产物;大自然终于实现了她的创造目标。人的智力是大自然所能产生的、难度最大的,同时又是最完美的作品。尽管如此,人与人之间在智力方面却表现出许多明显的梯级差别,只有极少数人能够具备最高级的智力。因此,从狭隘和严格的意义上说,极少数人所具备的最高级的智力是大自然创造的难度最大、等级最高的作品;同时,也是这个世界至为罕有、价值至昂之物。拥有如此高度智力的人,头脑具备了至为清晰的意识。世界在他的意识里面得到了清晰、完美的反映。因此,这种得天独厚的人也就拥有了这世界上最高贵、最具价值之物,他们也就拥有了快乐的源泉。与他们的快乐相比较,其他别的快乐简直就是微不足道的。这种人除了向外在世界要求得到闲暇以外,别无其他。有了闲暇时间,他们就能在不受外界打扰的情况下,精心呵护、擦拭自己的宝物,享受自己的这一份拥有。其他并不属于思想智力方面的快乐都是低级的,这些快乐只会引起意欲的活动,亦即引导人们进入希冀、欲望、恐慌和争斗之中。不管意欲朝着何种方向活动,它都不会不带痛苦地全身而退。另外,一般来说,随着意欲达到了它的目的,我们的失望也就出现了。但伴随着领略思想智力的快乐,我们体会到的只是更加清晰的真理。在思想智力的王国里,认知的活动,而不是痛苦,成为这里的主宰。要领略思想智力的快乐却必须自身拥有智力。并且,一个人所获得的这方面的快乐程度也是根据他的智力程度而定的,因为“世上的精神智慧对于一个没有精神智慧的人来说,几乎等于零”(拉布吕耶尔,1645~1696年,写讽刺作品的法国道德学家,著有《品格论》。——译者)。不过,拥有卓越的精神思想所带来的一个确切的不便之处,就是一个人感受痛苦的能力也伴随着他的智慧而增强了;在那些智力优越的人身上,所感受到的痛苦也达到了最高级。——原注 。}。要使一个凡夫俗子对某事物产生热切的关注,只能通过刺激他的意欲,并由此提起他对这事物的切身兴趣。但是意欲持久的兴奋,却不是单一纯净、不含杂质的,它是与痛苦紧密相联。在上流社会流行的纸牌游戏就是这样一种旨在刺激意欲的手段。的确,它能激发起人们肤浅的兴趣,但它带给人们的也只是暂时的、轻微的、而不是永久和严重的痛苦。正因为如此,我们只能把纸牌游戏视为对意欲的搔痒式的挑逗\footnote{根本上,平庸就是由于在人的意识里面,意欲完全地压倒了认识力,以致达到了这样的程度:认识力完全地服务于意欲。当意欲不再需要认识力的效劳时,亦即不存在或大或小的动因时,认识力就完全停止发挥作用了,这样,人的思想就呈现一片空白。但是,欠缺认识力的意欲是至为普遍的情形,它导致了平庸的状态。在平庸的状态中,只有人的感觉器官和处理感觉材料所需要的微弱理解力才保持活跃。因此,平庸的人每时每刻都全方位地接收所有印象,也就是说,他会眼看耳听所有发生在他身边的事情,甚至最微弱的声响和最微不足道的事情都会立即引起他的注意,就像动物的情形一样。这种平庸形于一个人的外在;从他的脸上和整个身体外部都可以看得出来。通常,完全占据一个人的意识的意欲越低级、自私和彻头彻尾的卑劣,那这个人的外观给人留下的印象就越令人反感。——原注。}。相比之下,具有优越精神能力的人却能够最热切地全情投入到认知活动中去,这里面不夹杂任何意欲的成分。事实上,他们这样热切投入也是迫不得已的事情。在他们全情投入其中的领域里,痛苦是陌生的。我们可以说,他们置身于神灵轻松自在地生活的地方。所以,大众的生活把大众引向浑噩、呆滞,他们的思想和欲望无一不是指向维护他们的个人安逸的那些渺小事务,正因为这样,他们的生活也就迈向了形形色色的苦难。所以,一旦他们停止为这些目标操劳,并且不得不返回依赖他们的自身内在时,无法忍受的无聊就向他们袭来。这时候,只有情欲的疯狂火焰,才可以活动一下那呆滞和死气沉沉的众生生活。但精神禀赋卓越的人却过着思想丰富、生气勃勃和意味深长的生活;有价值和有兴趣的事物吸引着他们的兴趣,并占据着他们的头脑。这样,最高贵的快乐的源泉就存在于他们的自身。能够刺激他们的外在事物是大自然的杰作和他们所观察的人类事务,还有那各个时代和各个地方的天才人物所创造的为数众多、千姿百态的杰作。只有这种人才可以真正完全地享受到这些杰作,因为只有他们才充分理解和感受到它们。因此,那些历史上的杰出人物才算是真正为他们活着,前者其实在向这些人求助了。而其他的人则只是偶然的看客,他们只是部分地明白个中的东鳞西爪。当然,具有天赋的人比常人多一个需求,那就是,学习、观察、研究、默想和实践的需求。因此,这也就是对闲暇的需求。但是,正如伏尔泰所正确无误地说过的,“只要有真正的需求,才会有真正的快乐。”所以,有这样的需求就是这些人能够得到别人所没有的快乐的条件。而对于其他人来说,尽管他们的周围存在各种各样大自然的美、艺术的美,以及思想方面的杰作,但是这些东西从根本上对于他们就像艳妓之于年老体衰的人。因此,一个具有思想天赋的人过着双重的生活,一种是他个人的生活,另一种则是思想上的生活,后者逐渐成为了他的唯一目标,而前者只是作为实现自己目标的一种手段而已。但对于芸芸众生来说,只有浅薄、空虚和充满烦恼的生存才会被视为是生活的目标。精神卓越的人首要关注的是精神上的生活。随着他们对事物的洞察和认识持续地加深和增长,他们的生活获得了一种整体的统一;精神生活的境界稳步提升而变得完整、美满,就像一件逐步变得完美的艺术品。与这种精神生活相比,那种纯粹以追求个人自身安逸为目标的实际生活则显得可悲——这种生活增加的只是长度而不是深度。正如我已经说过的,这种现实生活对于大众就是目的,但对于精神卓越者而言,那只是手段而已。 

我们的现实生活在没有受到情欲的驱动时会变得无聊和乏味;一旦受到情欲的驱动,很快就会变得痛苦不堪。因此,只有那些思想禀赋超常的人才是幸运的,他们的智力超出了意欲所需要的程度。只有这种人才能够在过着实际生活的同时,还享有一种不带痛苦的精神生活。他们全副身心地沉浸在这种精神生活当中,乐此不疲。仅仅拥有闲暇,即智力不需要为意欲服务,并不足以使人们享有精神生活;为能享有精神生活,人们必须具备某种真正充裕有余的能力。只有具备了这种充裕有余的能力,才能有资格从事并不服务于意欲的纯粹精神上的活动。相比之下,“没有精神思想消遣的闲暇就是死亡,它就像要把人活生生地埋葬”(塞尼加语)。根据各人精神思想能力参差不一的充裕程度,而相应在现实生活的同时,还有着无数等级的思想生活:从仅仅只是收集和描绘昆虫、鸟类、矿物、钱币之类的精神乐趣,一直到创作出最杰出的文学和哲学作品。类似的精神生活使我们得以避免低劣的社交,以及许许多多的危险、不幸、损失和纵欲。如果人们完全是在这个现实生活里追求幸福,那他们就会遭遇上述这些不好的东西。所以,例如,虽然我的哲学并没有给我带来具体的好处,但它却使我避免了许多的损失。 

但是,常人却寄希望于身外之物,寄望于从财产、地位、妻子、儿女、朋友、社会人群那里获取生活快乐;他把自己一生的幸福寄托在这些上面。因此一旦他失去了这些东西,或者对这些东西的幻想破灭,那他的幸福也就随之烟消云散了。为把这种情形表达清楚,我们可以这样说:这个人的重心在他的自身之外。正因为这样,常人的愿望和念头总是不停地转换。如果能力允许他这样做,他就会变换着花样,购买乡村别墅或者良种马匹;一会儿举行晚会,一会儿又出外旅游。总之,他要极尽奢华的享受,这是因为他只能从外在出发寻找得到满
足,这就像重病人一样,冀望通过汤水和药物重获身体的健康和力量。其实,一个人自身的生命力才是身体力量和健康的源泉。我们并不马上讨论处于对应的另一极端的人,我们首先看看那些精神思想力量并不那么显著突出、但却又超越了泛泛之辈的人吧。我们可以看到:当缺少外在的快乐源泉,又或者,当那些外在的快乐渠道再也无法满足他们的时候,这一类人就会学习和练习某一门优美的艺术,或者进行其他的自然科学的学习。例如:研究植物学、矿物学、物理学、天文学、历史学等等,并从中得到消遣和乐趣。对于这样的人,我们才可以说,他们的重心是部分地存在于自身。但是,这些人对艺术的业余爱好与自发的艺术创造力之间仍然存在一段相当的距离;又因为单纯的自然科学知识只停留在事物表面现象之间的相互关系,所以,他们无法全副身心投入其中,被它们所完全占据,并因此整个的生命存在与这些东西紧密地纠缠在一起,以至于对除此之外的任何事物都失去了兴趣。只有那些具有最高等的精神禀赋、我们称之为“天才”的一类人才会进入这样的状态,因为只有这些人才会把存在和事物的本质,完全而又绝对地纳入他们的课题。在这以后,他们就尽力把自己的深刻见解,以适合自己个性的方式,或通过艺术,或通过哲学表达出来。因此,对于这一类人来说,不受外界的打扰,以便忙于自己的思想和作品,实在已经成为迫切的需要。孤身独处正是他们求之不得的,闲暇则是至高无上的赐予。其他别的一切好处都是多余的——如果它们存在的话,那通常只会变成一种负担。只有这种人我们才可以说:他们的重心就在他们的自身当中。由此,我们可以解释清楚为何这类极其稀罕的人物,就算他们有着最良好的性格脾性,也不会对朋友、家庭和集体表现出其他人都会有的那种强烈的休戚与共的兴趣。他们拥有自身内在,那么,尽管失去了其他的一切也能得到安慰。因此,在他们身上有着一种孤独的特质;尤其在别人从来没有真正完全地满足过他们的时候,这种特质就更加明显。他们因而无法视别人为自己的同类。的确,当彼此的差异无处不在的时候,他们也就慢慢地习惯了作为另类的人生活在人群当中。在称呼人群时,他们脑子里想到的是第三人称的“他们”,而不是第一人称的“我们”。 

由此看来,那些在精神思想方面得到大自然异常慷慨馈赠的人,也就是最幸运的人了。确实,属于主体的东西比起属于客体的东西距离我们更近;如果客观事物真要发挥什么作用的话,无论其作用为何,那永远都是首先通过主体才能发挥作用。因此,客观事物只是第二性的。以下这些优美的诗句可以作证: 

 
\begin{quotation}
真正的财富只能是灵魂的内在财富; 

其他别的东西带来烦恼多于好处。 

——卢奇安语 
\end{quotation}


一个具有丰富内在的人对于外在世界确实别无他求,除了这一具有否定性质的礼物——闲暇。他需要闲暇去培养和发展自己的精神才能,享受自己的内在财富。他的要求只是在自己的一生中,每天每时都可以成为自己。当一个人注定要把自己的精神印记留给整个人类,那么,对这个人就只有一种幸福或者一种不幸可言——那就是,能够完美发掘、修养和发挥自己的才能,得以完成自己的杰作。否则,如果受到阻挠而不能这样做,那就是他的不幸了。除此之外的其他别的东西对于他来说都是无关重要的。因此,我们看到各个时代的伟大精神人物都把闲暇视为最可宝贵的东西;因为闲暇之于每个人的价值是和这个人自身的价值对等的。“幸福好像就等同于闲暇”,亚里士多德这样说过。狄奥根尼斯告诉我们:“苏格拉底珍视闲暇甚于一切”。与这些说法不谋而合的是,亚里士多德把探究哲学的生活称为最幸福的生活。他在《政治学》里所说的话也跟我们的讨论相关联;他说:“能够不受阻碍地培养、发挥一个人的突出才能,不管这种才能是什么,是为真正的幸福。”歌德在《威廉·迈斯特》中的说法也与此相同:“谁要是生来就具备、生来就注定要发挥某种才能,那他就会在发挥这种才能中找到最美好的人生。”但拥有闲暇不仅对于人们的惯常命运是陌生的、稀有的,对于人们的惯常天性而言也是如此,因为人的天然命运就是他必须花费时间去获得他本人以及他的家人赖以生存的东西。人是匮乏的儿子,他并不是自由发挥思想的人。因此,闲暇很快就成了普通大众的包袱。的确,如果人们不能通过各种幻想的、虚假的目标,以各式游戏消遣和爱好来填塞时间,到最后,闲暇就会变成了痛苦。基于同样的原因,闲暇还会给人们带来危险,因为“当一个人无所事事的时候难以保持安静”是相当正确的。但是,在另一方面,一个人拥有超出常规配备的智力却也是反常的,亦即违反自然的。如果真的出现这样一个禀赋超常的人,那么,闲暇对于这一个人的幸福就是必不可少的——尽管闲暇对于他人来说只是一种负担和麻烦。因为缺少了闲暇,这种人就犹如被套上木轭子的柏加索斯那样闷闷不乐。但如果上述的两种特殊反常的情形碰巧结合在一起一拥有闲暇属于外在的特殊情形,而具有超常禀赋则是内在的反常情形——那就是一个人的一大幸运。因为这样的话,那个得天独厚的人现在就可以过上一种更加高级的生活,也就是说,这样的生活免除了人生两个对立的痛苦根源:匮乏和无聊。换句话说,他再不用为生存而忧心忡忡地奔忙,也不会无力忍受闲暇(闲暇也就是自由的生存)。人生这两种痛苦也只有通过它们的彼此抵消和中和,才使常人得以逃脱它们的困扰。 

虽然如此,我们却要考虑到:一个具有优异禀赋的人由于头脑超常的神经活动,对形形色色的痛苦的感受力被大大加强了。另外,他那激烈的气质——这是他拥有这些禀赋的前提条件——以及与此密切相关的对事物和形象的更加鲜明、完整的认识,所有这些都使被刺激起来的情绪更加强烈。一般而言,这些感觉情绪总是给这种人带来痛苦多于愉快。最后一点就是巨大的精神思想禀赋使拥有这些禀赋的人疏远了他人及其追求。因为自身的拥有越丰富,他在别人身上所能发现得到的就越少。大众引以为乐的、花样繁多的事情,在他眼里既乏味又
浅薄。那无处不在的事物均衡互补法则或许在这里也发挥着作用。确实,人们经常挂在嘴边的,并且似乎不无道理的说法就是:头脑至为狭窄、局促的人根本上就是最幸福的,虽然并没有人会羡慕他们的这一好运。我不想让读者先入为主,在这一问题上给予一个明确的说法,尤其是索福克勒斯本人在这一问题上就表达过两种互相矛盾的意见: 

\begin{quotation}
头脑聪明对于一个人的幸福是主要的。 
\end{quotation}


又 

\begin{quotation}
要过最轻松愉快的生活莫过于头脑简单。 
\end{quotation}

 

在圣经《旧约》里,贤哲们的说法同样令人莫衷一是: 

\begin{quotation}
愚人的生活比死亡还要糟糕。

越有智慧,就越烦恼。 
\end{quotation}


在这里,我不会忽略提及这样的一类人:他们由于仅仅具备了那常规的、有限的智力配给,所以,他们并没有精神思想上的要求,他们也就是德语里的Philistine——“菲利斯坦人”。这名称源自于德国的大学生词汇。后来,这一名称有了更深一层的含义,虽然它和原来的意思依然相似;“菲利斯丁人”指的是和“缪斯的孩子”恰恰相反的意思,那就是“被文艺女神抛弃的人”。确实,从一个更高的角度审视,我应该把菲利斯丁人的定义确定为所有那些总是严肃古板地关注着那并非现实之现实的人。不过,这样一个超验的定义却跟大众的视角不相吻合——而我在这本书里所采用的就是大众的视角——所以,这样的定义或者不会被每一个读者所透彻理解。相比之下,这名称的第一个定义更加容易解释清楚,它也详细表现了菲利斯丁人的特质及其根源。因此,菲利斯丁人就是一个没有精神需求的人。根据我提及过的原则,“没有真正的需求也就没有真正的快乐”就可以推断:首先,对于他们的自身,菲利斯丁人并没有感受精神上的乐趣。他的存在并没有受到任何对知识的追求和对真理的探索这一强烈欲望的驱动,也没有要享受真正的美的热切愿望——美的享受与对知识、真理的追求密切相关。但如果时尚或者权威把这一类快乐强加给他们,那他们就会像应付强制性苦役般地尽快把它们打发了事。对这种人来说,真正的快乐只能是感官的快乐。牡蛎和香槟就是他们生存的最高境界。他们生活的目的也就是为自己获得所有能为他们带来身体上安逸和舒适的东西。如果这些事情把他们忙得晕头转向,那他们就的确快乐了!因为如果从一开始就把这些好东西大量提供给他们,他们就会不可避免地陷入无聊之中,而为了对抗无聊,他们是无所不用其极的:舞会、社交、看戏、玩牌、赌博、饮酒、旅行、马匹、女人等等。但所有这些都不足以赶走无聊,因为缺少了精神的需求,精神的快乐也就是不可能的。因此,菲利斯丁人都有一个奇异的特征,那就是:他们都有一副呆滞、干巴巴的类似于动物的一本正经和严肃表情。没有什么事情能使他们愉快、激动,能提起他们的兴趣。感官的乐趣很快就会烟消云散。由同样的菲利斯丁人所组成的社交聚会,很快就变得乏味无聊,纸牌游戏到最后也变得令人厌倦。不管怎样,这种人最终还剩下虚荣心。他们以各自不同的方式享受虚荣心所带给他们的乐趣,那就是:他们尽力在财富或者社会地位,或者权力和影响力方面胜人一筹,并藉此获得他人对自己的尊崇。又或者,他们至少可以追随那些拥有上述本事的人,以沐浴在这些人身上折射出来的光辉之中。从我们提到的这些菲利斯丁人的本质,可以引出第二点:对于他人,由于菲利斯丁人没有精神上的需求,而只有身体上的需要,所以,他们在与他人的交往中,会寻求那些能够满足自己身体上的需要,而不是精神上的需求的人。因此,在他们对别人的诸多要求当中,最不重要的就是一个人所具有的精神思想。当他看见别人具有突出的精神思想时,那反而只会引起他的反感,甚至憎恨。因为,他有着一股可憎的自卑感,以及呆笨的、不为人知的嫉妒心——他小心翼翼地试图把它们掩饰起来,甚至对自己也是这样。但这样一来,这种嫉妒有时候就会变成某种私下里的苦涩和愤怒。因此,他永远也不会想到要对卓越的精神思想给予恰如其分的尊崇和敬意;他一心一意地把尊崇和敬意留给拥有地位、财富、权力、影响的人,因为这些东西在他的眼中才是真正优越的东西。在这些方面出风头也就成了他的愿望。所有这一切都源于这一事实:他是一个没有精神需求的人。 

菲利斯丁人的巨大痛苦就在于任何理念性的东西都无法带给他们愉快。他们为了逃避无聊,不断需要现实性的事物。但由于现实性的东西很快就会被穷尽,一旦这样,它们就不但不再提供快乐,反而会使人厌烦。并且,这些东西还会带来各种祸殃。相比较而言,理念性的东西却是不可穷尽的,它们本身既无邪也无害。 


在关于何种个人素质、禀赋能给人带来幸福的所有这些讨论中,我关注的主要是人的体质和智力上的素质,至于人的道德素质以何种方式直接地给人以幸福——这问题我在我的关于道德的基础的获奖论文里面已经谈论过了。因此,我推荐读者阅读那篇论文。 


\chapter{人所拥有的财产} 
伟大的幸福论教育家伊壁鸠鲁正确而美妙地把人的需要划分成为三类。第一类属于人的天然的和迫切的需要。这类需要如果得不到满足,就会造成人的痛苦。顺理成章这一类需要也就是食品和衣物,它们比较容易得到满足。第二类需要同样是天然的,但却不是迫切的。那就是满足性欲的需要,尽管伊壁鸠鲁在《赖阿特斯的报道》中没有把它说出来(在这里我把他的学说表达得更清楚、更完整)。要满足这一类需要就相对困难一些了。第三类的需要则既不是天然的,也不是迫切的,那就是对奢侈、排场、铺张和辉煌的追求。这些需要没有止境,要满足这些需要,也是非常困难的。 

在拥有财产的问题上,要给我们合乎理智的愿望界定一个限度,如果不是不可能,那也是一件很困难的事情,因为一个人在拥有财产方面能否得到满足并不由某一个财产的绝对数量所决定。这其实取决于一个相对的数量,也就是说,由一个人所期待得到的财产和自己已经实际拥有的之间的关系决定。因此,仅仅考察一个人的实际拥有毫无意义,这种情形就犹如在计算一个分数时,只计算分子而忽略了分母一样。当对某一样东西的要求还没有进入一个人的意识的时候,这个人完全不会感觉到对它有所欠缺。没有这样东西,他照样心安理得。但一个拥有百倍以上财产的人,只要他对某样东西产生了要求,而又得不到它,那他就会感到怏怏不乐。在这一方面,对于一些他认为有可能得到满足的要求,每个人都有他的视线范围。他的要求不会超出这一视线范围。处于他心目中的视线范围之内的具体之物一旦出现,而他又确信能够得到它,那他就会感到幸福。但是如果得到这具体之物存在重重困难,他根本就没有得到它的希望和可能,那他就会感觉不幸和痛苦。所有在他视线以外的东西,都不会对他产生任何影响。因此,穷人不会因为得不到巨大的财富而焦虑不安,但富人在计划失算落空的时候,不会考虑到自己已经拥有相当可观的财物,并以此安慰自己。财富犹如海水:一个人海水喝得越多,他就越感到口渴。这一道理同样适用于名声。我们在失去了财富或者安逸的处境以后,当我们挺住了最初的阵痛,我们惯常的心境与当初相比较,并没有发生很大的改变——这是因为:当命运减少了我们的财富以后,我们自己也就相应降低了我们的要求。在遭遇不幸时,上述过程的确是痛苦方分的;但这个过程完成以后,痛苦也就减少许多了,到最后甚至感觉不到了,因为伤口已经愈合了。反过来,如果交到好运,我们的期望的压缩机就会把期望膨胀起来,我们在这过程中就感受到了快乐。但是,这一欢乐并不会维持长久。当整个过程全部完成以后,那扩大了的要求范围已经被我们习以为常了;并且,与新的要求相比较,我们就会对目前的拥有不以为然了。荷马在《奥德赛》的第十八节表达了我这里所说的意思。这一节最后的两行是这样的: 


\begin{quotation}
凡夫俗子的情绪飘忽不定,

就像神、人之父所赐予的日子。 
\end{quotation}
 

我们之所以感到不满,原因就在于我们不断试图提高我们的要求,但同时,其他妨碍我们成功的条件因素却保持不变。 

对于像人类这样一个贫乏不堪、充满需求的物种,财富比起任何其他别的东西都得到人们更多的和更真诚的尊重,甚至崇拜,这是毫不奇怪的。甚至权力本身也只是获取财富的工具。不足为奇的还有:为了达到获取财富这一目的,一切尽可以抛开,一切都可以推翻。例如,在哲学教授手中的哲学就落得了这样的下场。 

人们的愿望首先指向了金钱,人们热爱金钱甚于一切,人们经常为此受到责备。但是,人们热爱金钱却是自然的,甚至是不可避免的。金钱就像永远不知疲倦的普鲁特斯,每时每刻都准备着变成我们那飘忽不定的愿望和变化多端的需要所要求的物品。任何其他物品只能满足一个需要,诸如食物之于饥饿的人,醇酒之于健康者,药物之于病人,皮毛之于冬季,女人之于小伙子等等。因此,它们都只是“服务于某一特定的东西”,它们的好处是相对的。唯独金钱才具备了绝对的好处。因为它并不只是满足某一具体的需要,而是满足抽象中的普遍
的多种需要。 

我们应把手头上的财富视为能够抵御众多可能发生的不幸和灾祸的城墙,这些财富并不是一纸任由我们寻欢作乐的许可证,花天酒地也不是我们的义务。如果一个人凭藉自己的某种天赋才能——不管这种才能是什么——从最初的一文不名到最终赚到可观的金钱,那他就会错觉地认为:自己的天赋才能是恒久不变的本金,他以此赚取的金钱只是本金的利息而已。因此,他不会把挣来的一部分金钱积累成为固定长久的本金,而是把挣来的钱随手花掉。这样,他们通常最终陷入贫困,因为如果他们的才能只能维持短暂的时间,例如:几乎所有从事优美艺术的人都属于这一类情况,那么,他们的天赋才能就有枯竭、耗尽的时候。又或者,他们挣钱的本事依赖某种环境和某种风气。这种环境、风气随后消失了,这样,他们的钱财收入也就停止了。手工制作者尽可以像我上面所说的那样花钱大手大脚,财来财去,因为他们不会轻易失去制作才能,他们也不会被助手、帮工的力气所替代。并且,他们的产品是大众需求的对象,所以不愁找不到销路。因此,这一说法是正确的:“掌握一门手艺,就是拿到了一个金饭碗。”各种类型的艺人和艺术家遭遇的情形却不一样。正因为这样,他们获取的
报酬是如此的优厚。他们所挣得的金钱因此应该变成他们的本金。但他们却把挣来的金钱只当作利息。这样,他们就走向了贫穷的结局。相比之下,继承了财产的人起码立刻就正确地认识到何为本金、何为利息。所以,他们之中的大多数人会尽力稳妥地维护自己的本金。事实上,如果可能的话,他们至少会把利息的八分之一存起来以应付将来的需要。因此,他们大多数人都生活得充裕、富足。我这里所说的并不适用于商人,因为对商人来说金钱本身就是挣取更多金钱的手段,是他们生财的工具。因此,尽管金钱是他们以汗水换来的,但他们仍然会试图以最佳的方式运用这些金钱,以保存和增加其资本。因此,这些人比起任何别的阶层的人都更懂得巧妙、适宜地运用金钱。 

在一般的情况下,那些经历过匮乏和贫穷的人,比较不那么害怕贫困,因此更加倾向于奢侈豪华。这是比较那些只是听说过贫困的人而言的。前者包括那些交上了某种好运,或者,得益于自己拥有的某一专门的特长——不论这特长是什么——从当初的贫困迅速达到了小康生活的人;后者包括出生并成长于良好家境的人。后者更加着眼于未来,因此他们比起前者过着更加节俭的生活。由此,我们可以得出结论:贫穷并不像我们所粗略看到的那样糟糕。不过在这一例子里,真正的原因或许是那些出生于富有家庭的人把财富视为必不可少,是构成唯一可能的生活的元素,就像空气般的不可或缺。因此他们就像保护自己的生命一样警觉地保护自己的财产。所以他们通常都有条不紊、小心谨慎、勤俭节约。相比之下,出生于世代贫困之家的人却把贫穷视为理所当然的事情。他们所继承得到的财富对于他们来说只是一种多余的东西,把财富用作享受或挥霍才够合适!一旦把钱财耗尽,他们仍然像以前没钱的时候那样生活下去,并且,还免除了一样烦恼哩!这就像莎士比亚说的那样: 

 
\begin{quotation}
乞丐一旦跨上了坐骑,就非得把马跑死为止。 

——《亨利五世》 
\end{quotation}


当然,这种人对自己的运气和能力都抱有坚定的和过分的信心,因为这两者都帮助他们脱离了贫困的境地。不过,他们的信心更多地是在他们的心里,而不是在他们的头脑里。因为他们和那些出身富裕的人并不一样,他们并没有把贫困视为一个无底深潭。他们认为,只要脚踏实地用力蹬上几脚,就能重新浮上水面。这一人性的特征可以解释为何出身贫穷的女子,比起为夫家带来丰厚嫁妆的富家女,通常更加挑剔、讲究和更加奢侈、挥霍,因为在大多数的情况下,富家出身的女子不仅仅带来了钱财;她们比起穷家女还有着一种更为热切的、得之于遗传的保护财产的愿望。不过,谁要是对此持有相反的意见,那他可以在阿里奥斯图的第一首讽刺作品里找到支持他的观点的权威说法。但约翰逊博士却赞同我的意见:“一个习惯于处理钱财的有钱女人,会小心翼翼地花钱。但一个在结婚以后才首次获得支配金钱权力的女人,会在用钱的时候大胆妄为,她简直就是大肆挥霍。”(《约翰逊的一生》,博斯威尔著)不管怎么样,我都要奉劝那些娶贫穷女子为妻的人不要让她们继承本金,而只是领取一份年金。他们尤其需要注意,不要把孩子的财产交到她们的手上。 

我在这里提醒人们谨慎保存挣来的或者继承下来的财产。我相信这样做并没有用我的笔写了些毫无价值的东西。如果一个人从一开始就拥有足够的财产,能够享有真正的独立自足,也就是说,可以不用操劳就能维持舒适的生活——甚至只够维持本人而不包括他的家人就行——那就是一种弥足珍贵的优越条件;因为这个人就能以此摆脱纠缠人生的匮乏和操劳,他也就从大众的苦役中获得了解放——而这苦役本是凡夫俗子的天然命运。只有得到命运如此垂青和眷顾的人,才是真正自由的人。这样的人才成为自己的主人,是自己的时间和自己的力量的主宰。每天早晨他就可以说上一句:“今天是属于我的”。因此原因,一个拥有一千塔勒年金的人与一个拥有十万塔勒年金的人相比较,两者之间的差别远远少于前者与一个一无所有的人之间的差别。如果祖传的家产落到一个具备高级精神禀赋的人的手里——这个人所要从事的事业跟埋头挣钱并不怎么对得上号——那么,这笔遗产就能发挥出它的最高价值,因为现在这个人受到了命运的双重馈赠,他尽可以为自己的天才而生活了。他能够从事别人无法从事的事情,创造出对大众都有益处,且又能给自己带来荣耀的东西。他以这种方式百倍地偿还自己欠下世人的债务。处于同样优越生活条件的其他人则可以通过从事慈善活动为人类作出贡献。相比之下,如果一个人继承了遗产,但却又不曾做出任何上述事情——哪怕他只是尝试这样做,或者只是做出了点滴的成绩——或者,他甚至没有试图细致地研究某一门学问,以支持和推动这门学问的发展;那么,这样的人就只是一个可耻的无所事事者。这种人也不会感到幸福,因为免除了贫穷只会把他引至人生的另一个痛苦极端——无聊。他受尽无聊的百般折磨。假如贫穷的处境使他有事可做的话,他反倒会生活得更加幸福。百无聊赖、无所事事很快就会把他引向奢侈挥霍,由此他就被剥夺了他那不配享受的优越条件。许多有钱人到最后沦为贫困,就是由于有钱
就挥霍殆尽,目的只不过是为了从压迫他们的无聊那里谋求片刻的喘息。 

但如果我们的目标是要在公职服务中达至高位,那就是完全另一回事了,因为为此目标我们必须赢得朋友、关系和受到别人的青睐;只能以此方式获得逐级晋升直至最高职位。这样,从根本上来说,一文不名地来到这个世界反而更好。尤其这个人没有显赫高贵的出身,但却具备了一定的才能。如果这个人是一个一无所有的穷光蛋,那反倒是他的一个真正优势,他也可因此获得别人的提携。因为每个人喜欢和寻找的就是别人的缺点和不足——这在人与人之间的谈话里面已经如此,在国家公务事业方面情况就更是这样。只有一个穷鬼才会对自己绝对的、彻底的、全方位的劣势达到所需要的深信不疑的程度,才会认识清楚自己的无足轻重和毫无价值。只有在这种情况下,这种人才会接连不停地向人弯腰致意,也只有他们的鞠躬才会深至九十度。只有这种人才能忍受一切,且一直报以微笑。只有他们才知道自己的奉献是完全没有价值的;只有他们才会扯高嗓门,或者用醒目的黑体字,公开把拙劣的文字作品捧为巨著——那些作者不是高高在他们之上,就是极有势力;也只有这种人才会摇尾乞怜。因此,只有他们才会在青年时期就已成为倡导下面这一不为人知的真理的人——这一真理由歌德通过这些字句向我们展示了出来: 

 
\begin{quotation}
任何人都不要抱怨卑鄙和下流,因为 

在这世上只有卑鄙和下流才是威力无比的。 
\end{quotation}
 

相比之下,从一开始就生活无忧的人,却大多难以管束。他们习惯于高视阔步,并不曾学会上述为人处世的艺术。或许他们具有某样能引以为傲的才能,但他们应该认识到这些才能与平凡庸俗、溜须拍马根本无法匹敌。最终,他们会看到身居比自己更高位置的人的平庸和低劣之处。此外,如果他们还遭受别人的侮辱和种种令人愤慨的事情,他们就会羞愧、茫然和害怕。这可不是在这个世界上生存的办法。相反,他们应该和勇敢的伏尔泰一道说出这样的话:“我们在这世上时日不多,不值得在可鄙的坏蛋的脚下爬行。”随便说上一句,令人遗憾
的是“可鄙的坏蛋”这一词可适用于这世上的很多人。因此,我们可以看到尤维纳利斯的诗句:

 
\begin{quotation}
在局促狭窄的屋子里,无从施展, 

要昂首挺胸已经非常困难。 
\end{quotation}


更适用于艺术表演的职业,而不大适用于其他世俗、钻营的人们。 

在“人所拥有的财产”这一章里,我并没有把妻子和儿女包括其中,因为与其说一个人拥有妻子、儿女,还不如说妻子、儿女拥有他。朋友反倒更应该被划入一个人的拥有物里面,但甚至在这问题上,拥有者也还是在某种程度上成为别人的拥有物。 

 

\chapter{人所展现的表象 }
我们所展现的表象——这也就是我们的存在在他人心目中的样子——通常都被我们过分看重,这是我们人性的一个特殊弱点所致,虽然稍作简单的思考我们就可以知道,他人的看法就其本身来说,对我们的幸福并非至关重要。因此,很难解释清楚为什么每当一个人看见自己似乎留给他人一个好评语时,他的心里就高兴,他的虚荣心也就受到了某种安慰。一只猫受到爱抚时,就会发出高兴的声音。同样,当一个人受到他人的称赞时,愉悦之情就不可避免地洋溢于脸上。只要某种赞扬在一个人所期望的范围之内,那么,尽管他人的赞扬明显地
虚伪不真,他仍然会很高兴。这种人尽管遭遇真正的不幸,或者就算幸福的两个主要来源——这些在前文已经讨论过了——相当贫乏枯竭,但他人的赞许仍会给他们带来安慰。令人惊讶的是,无论在何种情况下,如果他们想获得别人好评的雄心受到任何意义上和程度上的挫折,或者,当他们受到别人轻视、不敬、怠慢时,都肯定会难过、伤心,很多时候还会感受到深刻的创痛。只要荣誉感是建筑在这种特殊的人性之上,那么,它就是道德的代替品,就会有效地促使很多人做出良好的行为。但是,对于人自身的幸福而言,尤其是对于与幸福密切相关的平和心境和独立自主而言,这种荣誉感更多地产生出扰乱和不良的作用,而不是有益的效果。因此,从增进幸福的观点出发,我们应该抑制这一人性弱点;应该细致考虑和恰如其分地评估它的真正价值,尽量减低我们对待别人意见的敏感程度,无论我们在受到别人意见的爱抚抑或伤害时都应如此,因为这两者悬挂在同一根线上。否则,人们就只能成为他人的看法和意见的奴隶: 

 
\begin{quotation}
使一个渴求赞语的人闷闷不乐或者兴高采烈的话语, 

却是多么的无足轻重! 

——贺拉斯语 
\end{quotation}


正确评估自己的自身价值和看待别人对自己的看法,对我们的幸福大有裨益。我们的自身包括了我们生存时间所包含的全部内容,我们生存的内在成分,以及我们在“人的自身”和“人所拥有的财产”这两章里所讨论过的各种好处。所有这一切都在我们的头脑意识里发挥作用;而别人对我们的看法只在别人的头脑意识中产生效果,它是附带种种概念性的东西呈现在别人头脑中的表象。所以,别人的看法对于我们的确并不直接存在,而只是间接地存在——只要别人对我们的行为并没有受到这些看法的影响和支配。只有当别人的看法对某事某物产生了影响,从而使我们自身也因此受到影响的时候,别人的这些看法才值得我们考虑。除此之外,在别人的头脑意识里面所发生的事情,对于我们并不重要。并且,当我们终于清楚地了解到:在大多数人的头脑里面都是些肤浅的思想和渺小的念头;这些人目光狭窄,情操低下;他们的见解谬误百出、是非颠倒;——到了这时候,我们就会逐渐对他人的评论淡然处之了。另外,从我们的自身经历就可以知道,一旦一个人不必惧怕别人,或者当一个人相信自己说的话不会传到被议论的对象的耳朵时,他就会不时地以轻蔑的方式议论别人。只要我
们听一听一帮蠢人是如何带着轻蔑的口吻议论最卓越、伟大的人物,我们就更加不会对他人的看法耿耿于怀了。我们也就会知道,要是太过于看重别人的看法,那就是抬举他们了。 

不管怎么样,如果一个人并不曾通过我已经讨论过的前两项内在和外在财富获得幸福,而只是在这第三项的好处里面寻找快乐,也就是说:他并不从自己的真正自我,而是从自己在他人头脑中的表象那里寻求满意和快感,那他就是相当不幸的。因为归根到底,我们存在的基础,因此亦即我们幸福的基础,是我们的动物本性。因此,健康对于我们的舒适是最重要的,其次就是维持生存的手段,亦即不带操劳的收入。荣誉、地位、名声——尽管这些被很多人视为价值非凡——却不能够和关键性的好处相提并论,或者取代它们;在必要的时候,为了前两项的好处,我们应该不容置疑地放弃这第三项好处。因此原因,认识下面这一朴素道理,会对增进我们的幸福大有益处:每一个人首先是并且实际上确实是寄居在自身的皮囊里,他并不是活在他人的见解之中;因此,我们现实的个人状况——这种状况受到健康、性情、能力、收入、女人、孩子、朋友、居住地点等诸因素的决定性的影响——对于我们的幸福的重要性百倍于别人对我们的随心所欲的看法。与此相反的错误见解只会造成我们的不幸。如果有人大声疾呼“名誉高于生命”,这其实就等于说,“人的生存和安适是无足轻重的,他人如何看待我们才是首要的问题”。这无论如何都是一个夸张的说法,这一说法赖以成立的基础是这样一个简单的道理:要在这人世间安身立命,名誉——即他人对我们的看法——对于我们经常是绝对必需的。关于这一点我会回头作进一步的讨论。但我们看到:几乎所有的人毕生不息地奋斗,历经千难万险,最终的目标就是让别人对自己刮目相看。人们拼命追逐官位、头衔、勋章,还有财富,其首要目的都是为了获取别人对自己更大的敬意,甚至人们掌握科学、艺术,也是从根本上出于同样的目的。所有这些都只不过令人遗憾地向我们显示了人类的愚蠢已经达到多么厉害的程度。把别人的意见和看法看得太过重要是人们常犯的错误。这一错误或许根植于我们的本性;或者,它伴随着社会和文明的步子而产生。不管怎么样,它对我们的行为和事业都产生了超乎常规的影响并损害了我们的幸福。具体的例子林林总总:从惊恐、奴性地顾忌“别人将会怎么说呢?”一直到古罗马护民官维吉尼斯剑插女儿的心脏这一极端的例子。一些人为了身后的荣誉,不惜牺牲个人的财富、安宁、健康,甚至生命。这一错误给那些要统治人或者驾驭人的人提供了一个便利手段。所以,在各种训练人的手法当中,加强和培养荣誉感的做法占据了首要的位置。但对于我们的幸福——这是我们的目的——荣誉感却是完全的另一码事。我反倒要提醒人们不要太过于看重别人对自己的看法。但日常经验告诉我们,大多数人还是把别人对自己的看法视为头等的重要,他们关注别人的看法更甚于关注那些活动在自己头脑意识里面、因而与自身有着更加直接关联的事情。这样,他们把自然的秩序本末倒置,别人的看法好像就是他们的存在的现实部分,而自己意识中的内容则反倒成了自己存在的理念部分;他们把派生的和次要的东西看作首要的事情。他们在别人头脑中的形象比起自己的本质存在更令他们牵肠挂肚。这种把非直接为我们所存在的东西作为直接的存在来加以看重的愚蠢做法,人们称之为虚荣,以表示这种渴望、努力所具有的虚幻和空洞本质。同样,从上面的论述可以轻易看出:这种虚荣为了手段而忘记
了目的,它和贪婪同属一类性质。 

事实上,我们对于他人的看法的注重,以及我们在这一方面的担忧,一般都超出了任何合理的程度,我们甚至可把这视为一种普遍流传的,或者更确切地说,是人类与生俱来的一种疯狂。我们无论要做或者不做什么事情,我们首要考虑的几乎就是别人的看法。只要我们仔细观察就可以看出,我们所经历过的担忧和害怕,半数以上来自这方面的忧虑。它是我们那容易受伤的自尊心——因为它有着病态般的敏感——和所有虚荣、自负、炫耀、排场的基础。一旦不再担心和期望别人的看法,那奢侈、排场十之八九就马上销声匿迹。形形色色的荣誉、骄傲,虽然内容和范围各有各的不同,但却都建立在别人的看法这一基础之上。它们要求人们作出多大的牺牲啊!在孩童时期,荣誉感就初露端倪;在接下来的青年期和中年期,名誉、骄傲等变得更加明显;但到了老年,这方面的欲望却显现得至为强烈,因为到了老年,享受感官乐趣的能力已大为减弱,虚荣和自大就与贪婪一道瓜分了统治的地盘。虚荣心在法国人的身上表现得至为明显,因为,法国人的虚荣心带有特定的地方色彩,通常会演变成为离谱的野心、可笑的民族虚荣和恬不知耻的大吹大擂。但这样的做法反倒使自己的努力落空,法国人因此成了其他民族的笑料,并获得了“伟大的民族”这一绰号。我这里有一个突出的例子,可以特别说明那种关注别人看法的行为所具有的反常本质。在这里,适当的人物和当时的处境互相结合,成为反映这种根植于人性的愚蠢的一个绝妙例子,因为它让我们测量出这种异常奇特的行为动机所具有的强度。下面这一段文字摘自1846年3月31日《泰晤士报》上一篇关于对托马斯·韦斯执行死刑的报道。托马斯·韦斯是一个手工制作学徒,他报复谋杀了自己的师傅,“在执行死刑的那天早上,监狱牧师很早就来到犯人的身边准备为他服务。韦斯举止安静,对于牧师的劝告没有丁点儿的兴趣,相反,他心里唯一惦记着的事情,就是在那些目睹他结束自己可耻一生的群众面前,能够壮起胆子,表现出勇气。他成功地做到了这一点。在韦斯步行穿过大院向在监狱里搭起的绞刑架走过去的时候,他高声发话——以便让旁边的人能够听见:‘啊!正如多德博士所说的,我很快就要知道那一个伟大的秘密了!’当时,他被绑着臂膀,但他不用别人的搀扶就迈上了绞刑架的梯子。走上梯子以后,他向左、右两边方向朝观望者鞠躬。聚集在下面的人群对此举立即报以雷鸣般的赞许声。”这可真是一个绝妙的例子:一个人已经可以看见那令人毛骨悚然的死亡了,此身之后,将是那漫漫无涯的永恒。但此时此刻,他并不关心别的,只是专心一意地要给那群凑热闹的乌合之众留下一个好的印象!在同一年的法国,一个伯爵因为试图谋杀国王而被判处死刑。在审判过程中,他担忧着能否穿戴体面地出现在元老院。到执行死刑的时候,他为能否获准刮胡子而忧心忡忡。在从前,情况并没有什么两样,这点我们可以翻看马迪奥·阿莱曼为他的著名小说《古斯曼·德·阿尔法拉契》所写的引言。这本书的引言告诉我们:许多迷惘的罪犯都把应该完全用于拯救自己灵魂的最后时间花在撰写和默记一篇简短的演说辞上面——他们就打算站在绞刑架的梯子上面宣读这篇演说。从这
些特殊的例子我们可以看到自身的影子,因为极端的例子往往最清晰地说明事情。在大多数的情况下,我们的忧心、烦恼、操劳、愤怒、恐惧都确实与别人对我们的看法有关。所有这些都和上述那些可怜的罪人的所作所为同样荒谬。我们的嫉妒和憎恨也大都出自同一根源。 

很明显,要增进我们的幸福——它主要依赖我们平和与满足的心情——再没有比限制和减弱人的这种冲动更好的办法了。我们要把它限制在一个理智的、可以说得过去的程度——这或许只是现在的程度的五十分之一而已。能做到这一点,那我们也就把这永远作痛的荆刺从我们的肉里拔了出来。不过要做到这一点是很困难的:因为这与我们某种天然的、与生俱来的反常本性有关。“名声是智者们最后才放弃的东西”——塔西佗如是说(《历史》第4,6)。要杜绝这种普遍的愚蠢做法,唯一的办法就是明确认识到这种做法的愚蠢。为此目的,我们必须清楚:人们头脑里面的认识和见解,绝大部分都是虚假荒唐和黑白颠倒的。因此,这些见解本身并不值得我们重视。此外,在大多数情况下,别人的看法对我们不会造成真正的影响。再进一步说,别人的意见一般都不是悦耳动听的,谁要是听到别人背后说他的话,还有说话的那种语气,那他几乎肯定会非常生气。最后,我们要知道:甚至名誉本身所具有的价值也只是间接而非直接的。当我们终于成功地摒弃了这一普遍的愚蠢做法,那我们内心的安宁和愉快就会令人置信地增加。同样,我们的举止和态度会变得更加自信、踏实,更加真实和自然。隐居生活之所以对于我们的心绪宁静有一种特别良好的影响,其主要原因就在于我们不用生活在别人的视线里。这样,我们就用不着时刻担心别人对我们会有这样或者那样的看法,我们也就得以恢复真我。同样,我们就可以躲过许许多多真正的不幸;因为拼力追求纯粹观念性的东西——更正确地说,应该是别人的不可救药的愚蠢想法——会把我们引人不幸。我们就会更多地关注我们拥有的那些实在的好处,不受干扰地享受它们。不过,正如这一句希腊文所说的那样:“高贵的也就是难以企及的”。 

我在这里谈论的这一源自人类本性的愚蠢,生发了三根主要芽条:好胜、虚荣和骄傲。虚荣和骄傲之间的差别在于:骄傲就是确信自己拥有某一方面的突出价值,但虚荣则尽力让别人确信自己拥有某一方面的突出价值;在大多数的情况下,伴随着虚荣的还有这样一个隐藏着的希望:通过唤起别人的确信,能够使自己真的拥有这一份确信。因此,骄傲是发自内在的、直接的自我敬重;而虚荣则是从外在、因而是间接地努力试图获得这一自我敬重。因此,虚荣使人健谈,但骄傲却让人沉默。但是,虚荣的人应该知道:要获得自己梦寐以求的别人
高度的评价,那如果他保持沉默,而不是夸夸其谈——哪怕他的嘴里可以说出最美妙、动听的话语——他将更加容易和更有把握地达到目的。不是任何人想骄傲就能骄傲得起来,他顶多只能装扮成一副骄傲的样子。不过,他很快就会露馅,正如任何扮演某一虚假角色的人很快都会露出原形一样。因为只有对自己的突出长处和非凡价值有一种发自内在的、坚定的和不可动摇的确信的人,才可以真正骄傲得起来。他的这一确信或许是错误的,或者,这一确信只是基于一些外在和泛泛的优点,但这一点对于他的骄傲是无关重要的,如果这一确信真正、确实存在的话。正因为骄傲根植于确信之中,所以,骄傲就和一切知识一样,并不存在于我们的主观随意之中。骄傲的大敌——我是说它的最大的障碍——就是虚荣。虚荣就是博取他人的赞许,以便在这一基础上建立起对自己的良好评价,但对自己有一坚实的良好评价却已经是骄傲的前提条件了。 

骄傲一般都受到人们的抨击和诋毁,我怀疑这些抨击和诋毁首要来自于那些并没有什么值得自己骄傲的人。面对大多数人的恬不知耻和傲慢无知,无论哪一个人,只要他拥有某一方面的优点,就要把自己的优点记在心上,不要把它忘了。因为如果一个人善意地忽略自己的优点,在与他人的交往时一视同仁地看待自己和他人,那么,他人就会公开坦白地把他认定为就是这个样子。我把上述这点意见特别推荐给那些具备最高级的长处的人,也就是说,具备真正的、个人的长处的人,因为这些长处并不像勋章和头衔那样每时每刻地作用于感官,从而使人们记得住它。否则,“蠢猪反过来教导智慧女神”(西塞罗语)的例子就会活生生地上演。“跟奴隶开玩笑,奴隶就会对你不屑”——这是一句了不起的阿拉伯谚语。并且,不要拒绝贺拉斯的这一句话:“你必须强迫自己接受应有的骄傲。”谦虚是美德——这一句话是蠢人的一项聪明的发明;因为根据这一说法每个人都要把自己说成像一个傻瓜似的,这就巧妙地把所有人都拉到同一个水平线上。这样做的结果就是在这世界上,似乎除了傻瓜之外,再没别样的人了。 

最廉价的骄傲就是民族的自豪感。沾染上民族自豪感的人暴露出这一事实:这个人缺乏个人的、他能够引以为豪的素质。如果情况不是这样,他也不至于抓住那些他和无数百万人所共有的东西为荣了。拥有突出个人素质的人会更加清晰地看到自己民族的缺点,因为这些缺点时刻就在自己的眼前,但每一个可怜巴巴的笨蛋,在这世上没有一样自己能为之感到骄傲的东西,那他就只能出此最后一招:为自己所属的民族而骄傲了。由此他获得了补偿。所以,他充满着感激之情,准备不惜以“牙齿和指甲”去捍卫自己民族所特有的一切缺点和愚蠢。德国人没有民族自豪感,这为他们那为人赞赏的诚实提供了一份证明。但是,其中那些滑稽可笑地假装为德国民族感到骄傲的人却不是诚实的——这主要是那些“德意志兄弟”和民主党人干的好事。他们奉承恭维德国人民,以便把他们引入歧途。他们甚至说是德国人发明了火药;我可不同意这种观点。利希腾贝格提出过这样一个问题:“为什么没有多少人会去冒充德国人?如果一个人想提高身份,一般他会宁愿冒充法国人或者英国人。这是为什么?”此外,一个人的独特个性远远优于民族性,在一个人身上所显现的独特个性比起国民性更应受到多一千倍的重视。因为国民性涉及的是大众,所以,坦率地说,它并没有多少值得称道的东西。在每一个国家,人们的狭窄、反常和卑劣都以某种形式表现出来,这就是所谓的国民性。我们对某一民族的国民性感到厌恶以后,就转而称道另一民族的国民性,直到我们同样厌恶它了为止。每一个民族都取笑别的民族,他们的嘲笑都是对的。 



我们在这一章所说讨论的话题——我们在这世上亦即在他人眼中的表象——正如上面已经说过的,可分为名誉、地位和名声。 

在大众和菲利斯丁人的眼中,地位、头衔相当重要;它们在国家机器的运转中发挥着巨大的作用。但对于我们增进幸福的目的而言,只须寥寥数语就可把它们了结。地位的价值是世俗常规的,也就是说,虚假不实的;它的作用是要得到别人虚假的尊敬,这完全就是为芸芸众生而上演的一出闹喜剧。


勋章就是汇票,它提取的是大众的看法;它的价值由汇票签发者的信誉而定。勋章的颁发除了顶替金钱酬劳、为国家节省大量财政开支以外,同时还是一种相当实用、妥当的安排——前提是勋章的颁发必须公正、有选择性地进行。大众除了长有眼睛和耳朵以外,就再别无其他。他们尤其缺乏判断力,记忆力也不强。人们作出的很多成绩和贡献完全超出了他们的理解范围,某些成绩和贡献在当下瞬间会被他们理解并获得他们的喝彩声,但时间过后不久,他们就会把它们忘记。我觉得这一做法非常适宜,那就是:透过十字或者星形勋章时时处处向着大众高喊“这些勋章的佩戴者和你们并不一样,他是做出过贡献和成绩的!”但是由于过滥和不加思考地颁发勋章,勋章也就由此失去了它们的价值。因此,勋章的颁发应该谨慎小心翼翼地进行,这就犹如商人在汇票上签名一样。十字勋章上所刻的“Pour le mérite”是画蛇添足的一句话。每个勋章都应该是奖励功绩的,这是不言而喻的。 

讨论名誉要比讨论地位、头衔复杂和困难得多。首先,我们需要给名誉下一个定义。为此目的,如果我说:名誉就是外在的良心,而良心就是内在的名誉,那这个说法或许能够满足很多人。但这种解释华丽、花哨多于清晰、透彻。因此,我认为,客观上,名誉是他人对我们的价值的看法;主观上,则是我们对于他人看法的顾忌。由于名誉的这一主观特质,它常常会给注重名誉的人带来某种有益的影响,虽然这种影响绝对不是纯粹道德方面。 

只要一个人不是彻底堕落,那么,他就会有名誉感和羞耻感,他就会珍视前者。名誉感和羞耻感的根源如下。单独的个人能够做的事情寥寥可数,他只是一个被抛弃在荒岛上的鲁滨逊。只有当他生活在与其他人组成的群体里,他才能有所作为。人的意识得到了发展以后,才会认识到自己这种处境。一旦这样,他就会产生愿望,希望被别人视他为人类社会中的一个有用成员,一个有能力履行自己的男人角色的人,并由此能有资格去分享社会所带来的好处。要成为这样的人,他必须首先做好每一个人都需要做好的事情;其次,他需要完成处于他那独特位置上人们所要求他和期望他做好的事情。但同样,他很快就认识到问题的关键并不在于他认为自己是一个有用的人,而在于别人是否也这样认为。他要获取别人对自己良好看法的热切愿望,以及他对别人的看法的无比珍视,也就由此而来。这两者都源自人的这种内在感觉——人们名之为“名誉感”,或者“羞耻感”,这根据情况而定。当一个人知道自己马上就要失去他人的好评时,尽管他清楚自己是清白无辜的,或者他犯下的过失并不严重,但他还是脸红了,这正是名誉感或者羞耻感所使然。在另一方面,没有什么比确信得到了别人的好评更能增强一个人的生活勇气,因为别人的好评向他允诺:众人会联合力量给予他保护和帮助,凭藉这堵比他自身力量强大得多的防护墙,他就可以对抗生活中的灾祸和困顿。 

人与人之间存在着各种各样的关系,一个人必须在这多方面的关系中得到别人的信任,亦即获得别人对他的好的看法。由此产生了多种多样的名誉。人与人之间的关系首先就是你、我之间的关系,接下来就是履行承诺,最后就是男女两性间的关系。与这些关系相对应的就是公民名誉、政府官员的名誉、男性名誉和女性名誉。每种名誉又可再分成更细的条目。 

公民名誉涵盖最广泛的范围,它所根据的是这样一个前提:我们无条件地尊重每个人的权利,因此,我们不可以运用不公正和法律不许可的手段谋取自己的利益。这是人与人之间进行和睦交往的条件。一旦我们做出某一明显严重违背上述前提的行为,和因此遭受了判罚——当然,前提是判罚必须公正——公民名誉就离我们而去了。归根到底,名誉的根据是对这一点的确信无疑:一个人的道德性格是不会改变的;所以,仅仅一次的恶劣行径就可以确切显示:一旦同样的情况重现,这个人以后的行为都会带有同样的道德实质。这点可从英语的
character(性格)一词得到证明。Character一词也表示名誉、名声的意思。因此,名誉一旦失去就不可以复得,除非这名誉的失去是因为某种误会,例如,是因为他人的诽谤或者人们只是基于假象作出了判断。正因为这样,才有了对付诬蔑、侮辱、匿名诽谤文章的法律,因为侮辱性的谩骂,是一种马虎草率的诬蔑,它并没有可以陈述的根据。这一句希腊话很好地表达了这一含意:“谩骂就是随意草率的诽谤”——也就是说,谩骂的内容都是子虚乌有的东西。当然,谩骂他人的人表明自己无法拿出被谩骂者的真正、确实的过错;否则,他就会把这些作为前提交代出来,然后充满信心地把结论留给他的听众去完成。但他却不是这样做。他提供了结论,但却说不出前提。他只能托词说这样做只是为了简便。的确,公民名誉中的“公民”亦即“中产阶级”,但这种名誉无一例外地适用于社会中的各个阶层,甚至最高阶层。任何人都不可以放弃这种名誉。公民名誉是非常严肃的,每一个人都不应该对此掉以轻心。不管一个人是谁,他干的职业是什么,只要他破坏了诚和信,他也就永远地失去了诚和信,随之而来的苦果肯定在所难免。 

在某种意义上,名誉和声望相比,名誉的特性是否定的,而声望则具有一种肯定的性质。因为名誉并不意味着别人认为:某某人具有某种特别的、为他的主体所独有的品质;名誉只是说明:某某人并没有欠缺每一个人根据规则都必须具备的品质。因此,名誉只是表明这一个人不是一个例外。但声望却表明这个人是一个例外。声望是要去争取的,相比之下,名誉只需要保有就行了。根据这一点,缺乏声望就是默默无闻,它是否定的;但缺乏名誉则是耻辱,它是肯定的。我们可不能把名誉的否定性质与被动性质相混淆。恰恰相反,荣誉具有一种
完全主动的特征;它纯粹从人的主体出发,以主体的行为为基础,而不是根据别人的所为以及外在的遭遇所决定,这也就是斯多葛派所说的“有赖于我们的事情”。这一点,正是真正的名誉和骑士名誉或虚假名誉的区别标志,我们在下文很快就可以看到。只有通过诋毁侮辱才有可能从外面对荣誉实施攻击。对付这种攻击的唯一对策就是把这种诋毁予以公开,把诋毁者公之于众。 

老年人得到别人的尊重似乎是基于这样一个事实:虽然也假设年轻人预享有名誉,但那名誉还未经实践考验的;因此他们的名誉是信用贷款。但对于老年人来说,在他们的一生中他们已经通过自己的行为证明了自己仍然保有名誉。单就年龄和经验而言,这两者都不足以成为要求年轻人向老者表示敬意的充足理由,因为动物也可以达至一定的岁数,甚至一些动物的寿命远远超过人类的寿命;而经验也不过是对世事的发展有了更加深入的了解而已。但在世界各地,人们都要求青年人向老年人表示尊重。高龄所带来的衰弱要求人们给予老人更多的是照顾和体贴,而不是敬意。但值得注意的是,人们对于花白的头发有一种天生的、因此确实是本能的敬意。皱纹是人到老年的一个更加确切的迹象,但皱纹却一点也不会引起人们的敬意。人们不会说令人肃然起敬的皱纹,而总是说令人肃然起敬的白发。 

名誉的价值只是间接的;因为,正如这一章的开首已经讨论过的,只有当他人对我们的看法决定性地影响了他们对我们的行为的时候,——或者只是有时候这样——他人的看法才具备价值。但是,只要我们和他人生活在一起,他人对我们的看法就会影响到他们对我们的行为。在一个文明的国度,我们的财产和安全都有赖于社会,而且我们无论干什么事情都需要得到别人的帮助;别人也只有对我们有了信任以后才会跟我们打交道。所以,别人对我们的看法具有较高的价值,虽然这一价值总是间接而非直接的。我无法认定别人的看法会有直接的价值。西赛罗的说法与我这里的说法不谋而合,他说:“克里斯玻斯和第欧根尼谈起好的名声时说:美名自有它的实际用处;除却这些好处,我们根本不值得花费哪怕丁点儿的力气获取美名。我完全同意他们的说法。”同样,爱尔维修,在他的巨著《论精神》里面详尽地讨论了这一真理,他得出结论:“我们喜爱别人的敬重并非因为敬重本身,而只是因为人们的敬重所带给我们的好处。”既然手段并非比目的更重要,那么“名誉比生命还要宝贵”这一被人们大肆渲染的格言就是,正如我已经说过的,一种言过其实的说法。 

关于公民名誉我就谈到这里。公职名誉就是人们普遍认为:担任公职的人真正具备了所要求的素质,他在任何情况下都能一丝不苟地履行他的公务员职责。一个人在国家事务中越发挥重要的作用,亦即他的职位越高,产生的影响越大,大众对于他的与其职位相应的智力素质和道德品质的要求就越高。所以,官员拥有更高一级的荣誉,而显示他的这一荣誉的是他的头衔、勋章,以及他人对他毕恭毕敬的态度。根据同样的标准,一个人的社会地位也决定了相应特殊一级的荣誉,虽然大众在判断社会地位的重要性这一问题上存在能力上的问题。正
因为边洋,地位在显示荣誉方面被打了折扣。但是,人们总是把更高的荣誉给予承担和履行不一般责任的人,而并非普通市民。后者的名誉主要是基于构成名誉的那些否定的素质。 

公职荣誉还进一步要求担任公务职责的人,对其所担任的职位保持尊敬。这是出于对他的同僚和他的继任者的考虑。要做到这一点,他必须严格履行自己的职责。除此之外,他不可以对一切针对他个人或者他所担任职务的攻击听之任之;换句话说,他不能对诸如他没有严格执行其官员职责或者他那职位对公众的福祉无所建树等言论无动于衷。相反,他必须通过法律手段来证明这些攻击是不公正的。 

拥有公职名誉的人还包括为国家效力的人、医生、公校教师,甚至公校的毕业生,亦即每一个被官方宣布为具备资格从事某一精神思想方面的工作而自己又自愿承诺投身于这样的工作的人。一句话,所有这一类为公众服务的人亦都享有公务官员的荣誉。真正的军人荣誉也属于公职荣誉一类,这是基于这一事实:每一个自愿保卫自己国家的人实际上具备了勇气、力量和坚强等必需的素质;并且,他准备着誓死保卫自己的国家,不会为了这世上的任何东西而抛弃已经宣誓效忠的旗帜。在此,公职荣誉一词我采用了比常规广泛得多的含意;公职荣誉的常规含意只是公民对一般公职所具有的敬意而已。 

至于性别的名誉,在我看来,则需要我们进行更为仔细的考察,并应对其原则下一番究本寻源的工夫。这也将证实:所有种种的荣誉毕竟都是基于实用利益的考虑。就其本质而言,性别名誉可分为女性名誉和男性名誉;并且,从男、女双方各自的角度看,都可以把这种名誉完全地理解为“团队精神”。但女性名誉远比男性名誉重要,因为女性与异性之间所建立的关系在女性的生活当中至为重要。因此,女性名誉就是人们这样的认可:作为未出嫁的女孩子,她还从没有把自己献给过哪一个男人;而作为妻子,她只把自己献给她的丈夫。人们如此
认可的重要性是由下面的道理所决定的。女性从男性那里要求和期待一切她需要的和渴望的东西。男性则从女性那里主要地、直接地只要求得到一样东西。因此,双方必须作出这样的安排:男方可以从女方那里得到他要的那一样东西,但条件是他必须承担照顾女方一切的任务,以及双方的结合所生下的子女。所有女性的福祉都有赖于这一安排。要实施这一安排,女性就有必要团结起来,显示“团队精神”。这样,女性就要形成一个整体,紧密团结以对抗她们共同的敌人——男性,因为男性通过得之于自然的、优越的身体和思想力量,占有了人世间所有的好处。女性必须征服他们和俘虏他们;只有通过占有他们,女性才可以占有那些人世间的好处。为此目的,女性名誉的训诫格言就是:绝对不能和男人发生非婚姻关系的性行为。只有这样,才能够强迫男性结婚——这是他们的一种投降;只有通过这样做,女性才能得到保障。但要完美达到这一目的则只有通过严格执行上述的训诫规定。所以,全体女性都以一种真正的团队精神密切留意着其他女性成员是否恪守这一训诫。因此,每一个由于进行非婚姻的性行为而背叛了全体女性的女孩子,都遭到她的同一性别的人的排斥、驱逐,并被打上耻辱的印记;因为一旦这种行为成为普遍,女性的福祉就会受到破坏。这个女人也就此失去了她的名誉。女人们再也不可以跟她交往,人们躲避她,就像躲避一个发臭的人。与此相同的命运也降临在一个与男人通奸的女人头上,因为对于这个女人的丈夫来说,她没有遵守她与丈夫所签下的投降合约。由于出现这样的事情,男性会害怕不敢再签订这样的合约了。而女性的解救却全赖男性签订合约。除此之外,因为通奸的女人粗暴地破坏了自己的承诺,并且她的做法带有欺骗的成分,所以,她失去女性名誉的同时,也失去了公民的名誉。因此,人们会有这种带原谅的说法,“一个失足的女孩子”,但却不会说“一个失足的女人”。在前一种情况下,诱奸者可以通过以与那女孩结婚的方式使她重获清白;但在后一种情况,尽管通奸妇人离婚以后,那个通奸的男人也无法让她重获清白之身。在得到这一清晰的见解以后,我们看到女性名誉原则的基石就是一种有益和必需的、但却是经过精打细算、建筑于实际利益之上的集体精神。这样,我们就能知道这种女性名誉对于女性的存在所具有的巨大重要性。因此,这种名誉具有一种巨大的相对价值,但这价值却不是绝对的;它不是那种超越了生命和生命之目的,因而也只能以生命为代价去取得的那种价值。因此,我们无法为卢克利斯和维吉尼斯的那些夸张、流于悲闹剧的所为喝彩。所以,爱弥尼亚·加洛蒂的结局有某些令人反感的成分,以致我们离开剧场的时候,心情都相当糟糕。在另一方面,撇下女性名誉不提,我们却忍不住要同情《艾格蒙特》中的克拉森。把女性名誉的原则推至极端就会像很多事情那样,为了手段而忘记了目的。因此把女性名誉如此夸大也就是赋予了这种名誉一种绝对的价值,但女性名誉比起所有其他名誉都更具备一种相对的价值。事实上,我们可以说,女性名誉的价值只是一种习俗的常规意义上的价值。关于这一点,从托马修斯的《论情妇》一书中可以看到:在过去几乎所有的国家和时期,直至马丁·路德的宗教改革,纳妾在法律上是被允许和承认的。根据这一法律,小妾可以维持她的名誉;古巴比伦的米利泰庙就更不用说了。当然,也有一些国家的情形,使婚姻的外在形式成为不可能,特别是在一些天主教的国家。在那里,是没有离婚的。在我看来,对于统治者来说,如果他们拥有情妇,而不是与她们缔结不相匹配的婚姻,那这样的做法会更加合乎道德。因为出自这不匹配的婚姻的子女,在合法的继承者死了以后,就会提出继承的要求。因此,这种婚姻引起内战的可能性总会存在,尽管这种可能性很小。并且,这样的一种不般配的婚姻,也就是说,把所有外在情况置之度外而缔结的婚姻,从根本上就是对女人和教士作出的妥协——而对这两
类人我们都要小心尽可能地不要作出让步。我们还要进一步考虑到:国家里面每个男人都可以娶到自己心仪的女人,只有一个男人被剥夺了这一自然的权利。这个可怜的男人就是一国之君。他求婚的手属于他的国家,这只手只能为着国家的理由交付出去。并且,他是一个凡人,他也渴望能够随心所欲一回。因此,阻止或责备君王试图拥有一个情妇,是既不公正,也不感恩的行为,同时,也是狭隘的。当然,这个情妇可不能给国家统治施加任何影响。至于遵守女性名誉方面,从这个情妇的角度来看,她在某种程度上是一个普遍规则之外的人。这是因为她把自己给予了一个爱她、而她又爱的男人,但这个男人却永远不可能对她明媒正娶。一般来说,女性名誉所带来的众多血腥的牺牲——婴儿惨遭杀戮、母亲自杀——都显示出女性名誉原则并不纯粹出自天然。当然,一个女孩子违反法律把自己交给了男人,这样做是对她所属的整个女性性别的人犯下的失信行为,虽然这种信约是一种心照不宣、没有经过郑重宣誓所定下的。在一般情况下,这个女孩子的利益会受到自己行为的最直接的损害,因此原因,在这件事情上,她的愚蠢更甚于她的卑劣。 

男性名誉是由女性名誉引出的。这对立的另一方的团队精神,要求每一个缔结婚姻、亦即签订了对对立一方有利的投降盟约的男性,密切留意这一协定是否得以执行,以防止由于执行盟约的马虎、松懈使此协定失去其坚固性。而男人既然为这桩交易付出了一切,人们会确保他达到他进行这桩交易的目的,亦即他能独自占有这个女人。因此,男性名誉要求男性必须对其妻子破坏婚姻的行为感到愤慨,并至少要通过与她分离来惩罚她。如果他睁着眼睛容忍妻子的所为,他就将被整个男性社会打上耻辱的印记。不过,这种耻辱并不如蒙受失去女性
名誉的耻辱那么严重。相反,这只是一个小小的瑕疵,因为对于男性而言,他有众多其他主要的社会关系,与女性的关系的重要性只是次一级的。新时代两个伟大戏剧诗人分别两次把男性名誉作为其作品的主题:莎士比亚在《奥赛罗》和《冬天的故事》及卡尔德隆在《医生的荣誉》和《秘密的侮辱、秘密的报复》。除此之外,男性名誉要求只是对女人,而不是对这个女人的奸夫作出惩罚,对后者的惩罚则超出了需要。这一点证实了男性名誉源自于男性的团队精神。 

至此为止,我所考察了的种种不同形式和原理根据的名誉普遍存在于各个民族和各个时代,虽然女性名誉的原则根据经证实有过区域性的、短时间的些微改变。相比之下,一种荣誉却是完全有别于上述各处普遍存在的种种荣誉。关于这种荣誉,希腊人、罗马人都没有丝毫的概念。直至今天,中国人、印度人和穆斯林教徒对它也都同样知之甚少。这种荣誉最初产生于中世纪,并只在基督教的欧洲扎根落户。它现在只在一小部分人,即社会上等阶层的人和攀附他们的人当中发挥作用,这种荣誉就是骑士荣誉。因为它的原则与我们到此为止所讨论过的荣誉的原则截然不同,甚至完全相反——因为后者培养出有荣誉感的人,而前者却要人们守住荣誉的空名——所以,我特地把骑士荣誉的原则一一列举出来,这些原则是骑士荣誉不成文的惯例,也是反映这种荣誉的镜子。 

1.骑士荣誉并不是他人对我们的价值的看法,这种荣誉纯粹取决于他人是否表示出他们的看法。至于他人所表示出来的看法是否出自真心则无关紧要,更不用说这种看法是否确有根据了。因此,尽管他人对我们的生活方式会有不好的看法,他们也尽可以蔑视我们,但只要他们不敢把自己的看法说出来,那么,我们的荣誉也就毫发无损。反过来,尽管我们以自己的品质和行为强取得到他人对我们的高度尊重(因为这并不取决于他人的主观随意),但假如任何一个人——不管这个人是多么的卑劣和愚蠢——把他对我们的蔑视表示出来,我们的荣誉就受到了破坏;如果不做出补救,那我们就会永远地失去这种荣誉了。这种荣誉完全不是取决于他人对我们的看法,而只是取决于他人是否说出他们的看法。证明我这一说法的充足根据,就是他人可以收回自己说出的诽谤和侮辱的话语;在必要的情况下,他们对自己的话语作出道歉,这样,一切事情就好像根本不曾发生过似的。至于他人是否就此改变了那构成了对别人的侮辱的看法,或者为什么改变了看法——这些对整件事情一点都不重要。只要宣布当初的表示无效,那么,一切都会完好如初。因此,当之无愧地获得别人的尊敬不是骑士荣誉的目的,它采用威吓手段来强求得到它。 

2.骑士荣誉并不取决于一个人做了些什么,而是取决于别人对他做了些什么。此前我所讨论的种种荣誉,根据其原理,取决于我们说了些什么和做了些什么,但骑士荣誉却与此相反:它取决于随便任何他人的言论和行为。骑士荣誉掌握在他人之手,或者更应该说,掌握在他人之口。荣誉随时——只要随便任何一个人愿意抓住随时的机会——都会一去不复返,除非受到攻击的人通过我马上就要讲到的方式程序重新把这荣誉夺回来。但这样做却是冒着失去自己的生命、健康、自由、财产和内心安宁的风险。由此得出这样的结论:尽管一个人的
所作所为公正、无私和高贵,他的心至为纯净,头脑也卓越不凡,但只要随便任何一个人愿意去侮辱他,他就随时有可能失去他的荣誉。这个中伤者可能只是一个毫无价值的恶棍,一个愚蠢到极点的饶舌者,一个百无聊赖的浪荡子,一个赌徒——一句话,一个不值得我们去理会和计较的人。但他在中伤别人的时候,却未必违反了骑士荣誉的法则。在大多数情况下,正是这—类人喜欢干侮辱别人的事情。正如塞尼加所说的,“一个人越可鄙、越可笑,他就越喜欢摇舌中伤。”此话极正确。这种人最容易被我们上文所描绘的那种人所激怒,因为相反对立的两类人都憎恨各自对方;另外,看到别人有着明显的优势只会引起毫无价值的人的无声怨恨。因此,歌德说: 

 
\begin{quotation}
你为什么要抱怨你的敌人? 

这些人能够成为你的朋友吗? 

你的本性, 

就是对他们的一个永恒的指责。 
\end{quotation}


卑鄙拙劣的人应该感激这种骑士原则才对!因为他把他们和优秀的一类人拉至同一水平上。如果不是这样的话,他们将在任何方面都无法企及那些优秀的人。一旦有人诋毁这些优越的人,亦即把卑劣的品质强加在他们身上,那么,这个人的诋毁在当下就是一个客观真实的、有根据的评判,一个具有确定效力的法令。事实上,它以后永远都会是真实的和有效的,除非人们旋即用鲜血把这种诋毁和侮辱抹掉。这也就是说,如果被侮辱者容忍这一侮辱扣在自己的头上,那么,被侮辱者(在众人的眼里是个“荣誉之士”)就是侮辱者(这或许是一个最
低劣的人)所说成的那个样子了。这样,他就会遭到“荣誉之士”的彻底鄙视。人们躲避他如同瘟疫一样。例如:人们会公开拒绝参加他到场的一切社交场合,等等。我相信我能够确切把这种观念的源头清理出来。从中世纪开始一直到15世纪(根据C.G.冯·韦斯特的《德国刑法、德国历史文集》),在进行刑事诉讼的程序时,并不是由原告去证明被告的罪责,而是由被告去证明自己的清白。履行的程序就是在他人担保的情况下作出一番宣誓。为此目的他需要担保的人。这些人必须发誓确信被告人不会作出伪证。如果被告人没有这些人帮忙,
或者原告人不承认这些担保的人,那么,事情就交由上帝作出判决。在一般情况下这都是以决斗的方式进行。因为现在被告人是“带着耻辱的人”,他必须为自己洗清耻辱。从这里,我们可以看出这蒙耻的概念、决斗的整个程序的源头。甚至时至今日,这样的事情(除了免去宣誓)还在“荣誉之士”之间进行。这就解释了为什么“荣誉之士”会对说谎的指责极度愤怒,并会因此寻求血腥的报复。考虑到说谎本是每天司空见惯的事情这一事实,这种反应就相当令人奇怪了。但是,这种反应已经成为根深蒂固的了,尤其在英格兰。事实上,如果以死亡威胁那些指控自己撒谎的人,自己就必须在一生中从未撒过谎。因此,在中世纪的犯罪诉讼中,形式变得更加简短,被告人只要回应原告一句:“你在说谎。”事情就可交由上帝作出判决了。据记载,骑士荣誉有这一规矩:对于撒谎的指责必须诉诸武力解决。关于言语的侮辱,我就写到这里。但是,还有比言语侮辱更为严重的事情,它的严重程度甚至在我提及它的时候,也必须请求所有信奉骑士荣誉的人的原谅,因为我知道仅仅想想它就足以让这些人毛发直竖,因为它是地球上至为凶恶的东西,简直比死亡和诅咒还要糟糕。那就是一个人出手甩给另一个人一个嘴巴子或者动手殴打他。这可怕的出手攻击会彻底摧毁被打者的荣誉。其他对荣誉的破坏可以用流血来愈合,但回应这出手一击就只能以置对方于死地才能带来荣誉的彻底复原。 

3.骑士荣誉跟一个人的自身到底是什么,或者跟他的道德本性是否会改变等诸如此类的学究气问题统统无关。当一个人暂时失去了荣誉,或者当荣誉受到损害,只要他迅速采取行动,通过决斗这一万应灵丹,他就可以迅速和完全地恢复荣誉。但如果冒犯者出自非信奉骑士荣誉的阶层,或者,如果他只是第一次冒犯了别人的荣誉,那么,人们就可以采用一种安全妥当的方式。人们可以在他冒犯自己的当下把他击倒——如果手头有现成武器的话,必要时,在以后任何情况下击倒他也可以,尤其是对方以出手袭击的方式破坏别人的荣誉。不过
就算他对我们的冒犯只停留在言语上,情况也还是一样的了。只有通过这样的还击,荣誉才能挽回。但假如人们不想走出这一步,以避免这一步行动所带来的不良后果;或者,假如人们并不确定冒犯者是否受到骑士荣誉法则的约束,那么人们还有更妙的一招。那就是,如果冒犯者表现粗鲁的话,那就以更加粗鲁的方式予以迎击;如果口头侮辱已经无济于事了,那我们就要拳脚相向,这是挽救荣誉的最极端的手段。因此,脸上挨了对方一个耳光,就只能以棍子回应;而对付棍子的攻击则只能报以打狗的鞭子做补救;对付打狗的鞭子,有人建议不妨采用唾脸作为最绝的一招。只有当这些都无济于事的时候,我们才不得不采用流血的行动。采用这种解决办法的根据是如下的骑士荣誉的格言。 

4.正如蒙受别人侮辱是一种耻辱,同样,发出侮辱则是一种荣誉。假如真理、公正和理性站在我的对手的一边,但如果我出言侮辱他,那么真理、公正和理性就得卷起包袱滚蛋,而道理和荣誉也就站到我的一边来了。与此同时,对方也就失去了荣誉,直至他用枪击、剑砍,而不是用公正和理性,去重获他的荣誉为止。因此,在荣誉这一问题上,粗野无能可以取代、并且优于一切个性品质,道理站在至为粗野无礼的行为一边。干吗还需要多种多样其他别的?一个人尽管很愚蠢、很卑劣、有失教养,但粗野无礼的行为可以抹去这一切,使一
切合法化。当我们交谈或者讨论问题的时候,如果某一个人向我们显示出他比我们对所谈论的话题有着更精确的认识,他比我们更热爱真理;他的判断力和理解力更加健全和优越;他表现出来的优越的精神智力使我们相形见绌——那么,我们可以一举消除他的所有优势,以及因为他的优势的缘故我们所暴露出来的劣势和不足。我们甚至可以反过来变得优于这个人——只要我们撒野、动粗就可以达到这一目的。粗野蛮横更胜思想交锋一筹,它抹杀了人们的精神智力,所以,如果我们的对手不跟我们计较,不是采用更加粗鲁的方式作出回应,并由此卷进一场高贵的决斗,那么我们就可以成为胜利者,荣誉就归于我们了。面对那君王般威力无比的粗野蛮横,真理、知识、思想、理解和机智都退避三舍,被逐离战场。所以,一旦有人表示出不同的见解,或者显示出更胜一筹的智力,这些“荣誉之士”就随时跨上战马;如果在争论问题时缺乏论辩,他们就会搜索寻找粗野的话语——这同样可以帮助他们取胜,并且,寻找粗野的话语也更加轻松容易。这样,他们就得胜凯旋了。由此看出,人们称赞这种荣誉原则可以使社会格调变得高尚,他们是多么的正确。这一条格言是以下文为根据的——它是整个荣誉规则的根本和灵魂。 

5.在信奉骑士荣誉的人看来,在判断人们之间的争议何者为公理时,我们所能寻求的最高裁判庭就是我们的身体力量,也就是说,我们的动物性。每一种粗鲁行为的发生确实就是人们诉诸动物性所致,因为做出粗暴的行为亦即宣布了精神力量和道德正义的交锋已经无力解决问题;取而代之的就是人的身体力量的搏斗。富兰克林把人定义为“会制造工具的动物”,所以,人的身体力量的交锋就由人带上只有人才会制作的武器来进行;这也就是说决斗。通过决斗,人们得到了一个无法挽回的判决。“荣誉之士”的基本格言,众所周知,可由Faustrecht“拳头公理”一词表达;这种表达法与Aberwitz一词相类似,同样具有讽刺意味。所以,骑士荣誉应被称作“拳头的荣誉”。 

6.如果我们发现公民荣誉在对待人我的关系问题上小心谨慎、承担义务、履行诺言;那么,相比之下,我们现在所讨论的骑土荣誉规则,在处理上述人际关系时,则显示了最高贵的宽松。只有一样东西是不能打破的,那就是以荣誉的名义所说出的话,也就是人们在说了“以荣誉担保”之后所给予的承诺。由此就引出了这样的一个假设:其他的承诺都可以不必履行。在迫不得已的情况下,我们甚至可以打破以荣誉的名义许下的承诺。只要通过决斗这一万应灵药,就能够挽回我们的荣誉——决斗的对手就是那些坚持说我们是以荣誉的名义许下了承诺的人。另外,只有一种债务是我们务必偿还的——那就是赌债,因此赌债被称为“荣誉债”。至于其他的债务,人们尽可以像犹太人和基督徒那样互相欺骗,而不会损害我们骑士荣誉的分毫\footnote{这些就是骑士荣誉的规矩。一旦明白地说出来,并把它们整理为清晰之概念以后,这些骑士荣誉原则显得多么荒诞和滑稽。甚至时至今日,在基督教盛行的欧洲,不少人都普遍敬仰骑士荣誉,而这些人都是属于所谓的上流社会,属于所谓有品味的人。的确,这些人当中许多人从小就受到骑士荣誉的言传身教,他们相信这些原则更甚于基督教教义的问答手册。他们对这些原则最真诚的敬畏,并且随时准备着为它们奉献出自己的幸福、安宁、健康和生命。他们认为,这些原则扎根于人性,因此,是人们与生俱来的,是有其先验的基础的。所以,这些原则超乎调查、探究。我不想伤害这些人的心,但他们的头脑确实让人不敢恭维。所以,对于注定要代表这世上的智识、成为泥土中的盐巴,并且做好准备承担天降之大任的团体——亦即我们的青年学子,这些原则尤其不相适应。但不幸的是,在德国,这些青年学子敬仰这些原则更甚于任何其他阶层。这些学生从希腊、罗马的著作中受到教育(当我还是学生中的一员时,那个毫无价值的冒牌哲学家、至今仍被德国学术界尊为哲学家的约翰·费希特担任这些教学工作)。现在,我并不想向你们强调由这些原则引致的恶劣的、违犯道德的后果。我只想向你们说出下面这些话:年轻的时候就接受希腊、罗马语言和智慧的福音的你们——从早年起,人们就不遗余力地让你们年轻的头脑受到优美、古老的尊贵和智慧之光的照耀——你们愿意接受这套愚蠢和野蛮的东西,作为你们的行为准则吗?想一想吧,我已把这些原则的种种可怜的狭窄、缺陷以最清晰的方式呈现在你们的眼前,就让它们成为检验你们的脑,而不是你们的心的试验石好了。如果你们的头脑不把它们抛弃,那你们的头脑就不适宜在这些领域发挥;在这些领域需要具备轻松打破偏见枷锁的敏锐判断力和明辨真假的透彻理解力——甚至在真假之分深藏不露、并不像在这里那么容易把握的时候。既然这样,我的小伙子们,就尝试在其他途径获取荣名吧,参军或者学习一门金饭碗一般的手艺都行。——原注。}。不怀偏见的读者一眼就可以看出,这一奇怪、野蛮、可笑的荣誉规则并非源自人类的天性,也不是得之于对人际关系的健康的理解。这一点可以从骑士荣誉只在极为狭窄的地盘之内发挥作用得到证实。也就是说,它唯一流行在欧洲,并且那也只是从中世纪才开始的。在欧洲,骑士荣誉也只在贵族、军队和仿效他们的人当中有效。希腊人、罗马人、甚至古今高度发达的亚细亚民族都不知晓这种荣誉及其规则。他们都只懂得我开首分析的那些荣誉。因此,这些民族对一个人的看法全由这一个人的所作所为来决定,而不是听任随便一个饶舌者的随心所欲的摆布。他们都认定这一事实:一个人说的话或者做出的行为,只能贬损自己本人的荣誉,而不会对别人的荣誉构成损害。他们遭受的一记耳光对于他们而言也就只是一记耳光而已,一匹马或者一匹驴子甚至能够踢出更加危险的一脚呢。别人的出手攻击会刺激起挨打者的愤怒,这视当时的情形而定,并很有可能当场就由此招来回击。但这些跟荣誉并没有关系。人们肯定不会准备着一本账目,记录着挨了别人的某一攻击或者遭受了某一些侮辱性的言词,以及为此已经得到
的或者还不曾得到的报复“满足”。这些民族的英勇气概和视死如归的气节,并不逊色于欧洲的基督徒。希腊人和罗马人是真正的英雄;但他们对“骑士荣誉”却一无所知。对于他们来说,决斗并不是高贵的人所做的事情,只有角斗士、标价出售的奴隶、判了刑的罪犯才会去做这样的事情——他们交替和野兽搏斗,以博大众一乐。随着基督教的引入,角斗士的表演遭到了禁止,在这基督教的时代,取而代之的是人与人之间的决斗,决斗的结果就成了上帝的判决。如果角斗士的搏斗表演是为了大众的娱乐而做出的残忍牺牲,那么,这种决斗则是为了大众的谬见而付出的残忍牺牲;但这里做出牺牲的不是罪犯、奴隶和囚徒,而是自由人和贵族。 

大量得以保存下来的证据表明骑士荣誉的谬见对于古人来说是全然陌生的。例如,一个条顿族首领向马略下战书,要求与他进行决斗,但这个英雄却捎话给这个首领:“如果他(这个首领)厌倦了生活,他尽可以上吊了结此生。”当然,马略主动提出向这个首领提供一个退了役的角斗士,这样他就可以跟这个角斗士展开一番较量。在普卢塔克的书里,我们读到舰队统帅欧里比亚德斯在跟德谟斯托克利斯争论时举起了棍子要打他。但德谟斯托克利斯并没有拔出佩剑,他只是说道,“你打我吧,但你要听我把话说完”。雅典的士官团并没有立即宣布不再愿意在德谟斯托克利斯的手下卖力。读到这种事情,信奉骑士荣誉的读者该是多么的忿忿不平!因此,一个当代法国作家说得很正确,“如果竟然有人说,德谟斯芬尼是个执着于骑士荣誉的人,人们只能给予同情的一笑;同样,西塞罗也不是执着于这种荣誉的人”(C·杜朗,《文学之夜》1828,卷二)。另外,柏拉图著作中的一段讨论虐待的文字已经足够清楚地显示,古人在对待类似事情时,并没有骑士荣誉的概念。苏格拉底因为经常与人辩论而常常受到别人粗野的对待,但他却处之泰然。有一次他被人踢了一脚,他默默地忍受,并对露出惊讶神情的人说,“如果一只驴子踢了我,我也要生气报复吗?”(狄奥根尼斯)还有一次,有人问他:“那个人不是在羞辱你吗?”“不”,他回答说,“因为他说的人可不是我。”斯托拜阿斯为我们留下《穆索尼斯》的一大段文字,从这段文字我们可以看到古人如何看待遭到别人的侮辱。除了诉诸法律以外,他们并不知道还会有另外别的解决办法,明智的人甚至不屑于采用这一解决办法。古人挨了别人的一记耳光,只会采用法律途径寻回公道——这点见之于柏拉图的《高尔吉亚篇》。在这一章里,还可以读到苏格拉底对此发表的意见。类似的事实还见之于《吉里斯的报道》:某个叫做卢西斯·维拉图斯的人,在没有受到挑衅的情况下,竟然斗胆把在路上遇见的罗马市民都打了一记耳光。事后出于避免把事情复杂化的考虑,他要一个奴隶带着一袋钱币在前面引路,把相应的25阿斯的赔偿金马上支付给那些诧异莫名的人。大名鼎鼎的犬儒学派大师克拉特斯就挨过音乐家尼克德洛姆斯重重的一记耳光。他整个脸都带着血印肿胀了起来。克拉特斯就把一张写着“这是尼克德洛姆斯所为”的字条贴在额头上,借以羞辱这个笛子演奏家,因为他竟然野蛮地对待这个被整个雅典奉若神明的人。在色诺彼的狄亚根尼斯致梅里斯玻斯的一封信中,他说一帮喝醉酒的雅典青年用鞭子抽打了他一顿,不过,这对于他并不是什么大不了的事情。塞尼加在他的《永恒的智慧》一书中从第十章直到最后,详尽讨论了如何对待别人的侮辱这一论题。他得出这样的结论:一个智者用不着理会这些东西。他在第十四章写道,“一个智者受到了袭击,他应该怎样做?卡图挨了一记耳光以后,并没有发愁,没有去报复,没有表示原谅。他只是宣称并没有发生过挨打的事情。” 

“是啊”,你们说:“但这些人可是智者啊!”——那你们就是愚人了? 

的确如此。由此看出,古人根本不知骑士荣誉原则为何物。这是因为古人在各个方面都忠实于自然,不带偏见地看待事物。他们不会轻信这些不吉祥、不可救药的丑陋的东西。他们把打在脸上的一巴掌就只看作是别人打来的一巴掌,那会造成身体轻微损伤,除此之外,他们不会把这看作任何别的东西。但对于当代人来说,挨了别人的一记耳光将是一桩巨大的灾难,足以构成一出悲剧的主题,例如高乃依的《熙德》。还有最近一出描写市民生活的德国悲剧,名叫《环境的力量》,但它应该叫做《谬见的力量》才对。如果一个人在巴黎国
民议会厅挨了一记耳光,那么这个耳光的声音就会从欧洲的一头传到另一头。那些执着于骑士荣誉的人,看到我引据的那些古老的经典事例肯定会生气、不舒服。为了对症下药,我建议他们阅读狄德罗的名著《命运主义者雅克》里面关于德格朗先生的故事。这是描述固守现代骑士荣誉的一部出类拔萃的代表作。他们会喜欢这本书,并从中获得启发\footnote{两个信奉骑士荣誉的人追求同一个女子,其中一人名字是德格朗。这两个人并排坐在桌子旁边,面向这个女人。德格朗谈吐活泼,试图吸引这个女子的注意。但这个女子心不在焉,好像并没有倾听德格朗的说话,而是不时地瞟着德格朗的情敌。当时,德格朗手里正握着一只生鸡蛋。一股病态的嫉妒驱使他捏碎了这只鸡蛋。鸡蛋弄破了,并且溅在了他的情敌的脸上。他的情敌的手动了一下,但德格朗握住了他的手,小声地在他耳边说了一句:“我接受你的挑战”。在座的人陷入了一片静默。第二天,德格朗的右颧骨上围上了一块厚厚的黑石膏,他们决斗了。德格朗的对手遭到了重创,但伤势还不至于致命。德格朗的右颧骨上的石膏减小了少许。他的对手复原以后,他们又进行了第二次的决斗。德格朗弄伤了对手,他把石膏又弄去了一小块。如是发生了五、六次。每次决斗以后,德格朗就把石膏弄掉一点点,直到敌手终于被杀死为止。啊!这旧时代的高贵骑士精神!不过,认真说来,谁要把这一典型故事跟以往发生的这类事情对比一下,就一定会说,一如在其他的事情上面,古人多么伟大,现代人又多么渺小!——原注。}。从上面的讨论,我们已经可以清楚地看到:骑士荣誉的原则并没有独到的见解,它也不是建筑在人性的基础之上,它只是人为的产物,其产生的根源并不难找到。它诞生于一个特定的时代。在那个时代,人们更多运用的是拳头,而不是大脑;人们的理性被教士们用铁链禁锢了。所以,它是被颂扬的中世纪及其骑士制度的产儿。那时候,人们不仅需要上帝关怀他们,而且还要上帝为他们评判事物。因此,困难的法律案件就经由仲裁法庭,或者上帝的判决,来作出决定。这几乎无一例外地演变为双方的决斗。决斗并不只是在骑士之间进行,而且还在市民之间进行。莎士比亚的《亨利六世》(第二部分第二幕第二景)里面就有一个绝好的例子。获得法律判决以后还可以上诉,亦即求助于决斗——它是更高一级的法庭,是由上帝做出的判决。这样一来,身体的敏捷和力量,亦即动物本性,就取代理性坐在了法官的位置。它不是根据一个人所做的事情,而是他最后遭遇运气的结果,来做出是或非的判决——这种情形符合至今仍然生效的骑士荣誉原则。谁要是对决斗的这一根源还有所怀疑,那他就阅读J.G.梅林根的那本出色的《决斗的历史》(1849)一书吧。事实上在今天,在那些遵循这种荣誉原则生活的人当中——这些人通常都没怎么受过教育,对事情不作深思——确实还有一些人把决斗的结局看作是上帝对他们所争执的问题作出的判决。当然,这种看法是根据传统流传下来的观点形成的。 

这就是骑士荣誉的根源。此外,它倾向于通过身体力量的威胁,强行得到他人表面的尊重;真正下工夫去争取别人的尊重,会被认为是一件既困难、又多余的事情。奉行骑士荣誉的人就好像用手捏着温度计的水银球,希望随着水银的上升,他的房间也就会暖和起来。深入考察一番,就可以知道问题的关键在于公民名誉的目的是与他人进行友好交往;它的内容就是别人对我们这样的看法:我们是值得人们完全信任的,因为我们绝对尊重别人的权利。但骑士荣誉却立足于别人对我们这样的看法:我们是令人生畏的,因为我们会绝对地、无条件地保护我们的权利。本来,这一条原则——令别人生畏比享有别人的信任更加重要——并没有什么大错,因为如果我们生活在自然状态之中,每一个人都必须保护自己的安全和直接捍卫自己的权利,这样,我们是不能信赖人类的正义的。但在文明时期,国家承担了保卫我们人身和财产的任务,那么,上述的原则就没有用武之地了。它就像坐落在别致的农田、热闹的公路,甚至铁路之间的城堡和瞭望塔——它们都是从拳头即公理的年代所遗留下来的弃置无用的东西。顽固遵循这一原则的骑士荣誉对付的只是人们做出的一些轻微的越轨行为——对于这些行为,国家只是实施轻微的处罚,或者根据“法律不理会微不足道的事情”的原则,根本就置之不理。因为这些都只是芝麻绿豆性质的侮辱,有时只是纯粹的取笑打闹。但在处理这些事情方面,骑士荣誉却夸大了人的价值,达到了一种与人的本性、构造和命运都完全不相称的程度。人的价值被提高至神圣不可侵犯的地步。这样,人们就会认为国家对于那些小小的冒犯所给予的惩罚根本不足够;被冒犯者就要自己执行惩罚的任务,目标指向了冒犯者的身体性命。很明显,事情的根本原因是人的极度自大和那种令人反感的盛气凌人——人们完全忘记了人到底是什么。骑士荣誉苛求人们不犯任何过错,也绝不接受任何伤害。谁要是准备用武力执行这种见解,并且宣告这一格言:“凡是侮辱过我或者动手打我的人一定要死亡。”那么,他倒的确应该被人们驱逐出他的国家\footnote{骑士荣誉是自大和愚蠢的产儿。与骑土荣誉针锋相对的真理则由卡尔德隆的《永恒的原则》里的一句台词表达出来:“贫乏就是阿当的命运”。值得注意的是,这种极端的自大傲慢竟然独一无二地出现在信奉如此宗教的人们当中——这一宗教要求它的信徒们把表现最大的谦卑作为他们的责任;因为在这之前的世纪,在其他各大洲,都不曾听说过这种骑士荣誉的原则。但我们却不能把它归之于宗教的原因,而应该把它归于封建制度。在这种制度下,每个贵族都自视为小皇帝,不承认在他之上还会有由人担任的裁判者。所以,他把自己视为神圣不可侵犯。因此,每一针对他的侮辱言词和攻击行为就犹如十恶不赦的死罪。因此,骑士荣誉和决斗本来就是属于贵族的事情。后来,士官阶层的人也仿效了这种习气。士官们不时地和上层社会交往,以避免自己显得太不重要。当决斗从神裁那里发展出来时,决斗可不是原因,而是骑士荣誉的实行和发展的结果。不承认任何由人担任的判决者的人,会寻求上帝的裁决。不过,神裁可不是基督教所特有的,它在印度教也有很大影响力,尤其在古老时代。它的痕迹至今犹在。——原注。}。为了美化这一胆大妄为的自负,林林总总的一切都被用作了借口。如果两个不怕死的人相遇,不肯主动为对方让路,那么,从轻轻的推碰就会演变成恶语相向,然后就是动起拳脚,最后事情就以某一方受到致命的一击而告终。其实,干脆跳过中间的环节,马上就动用武器,这样更能保住颜面。细致具体的程序发展成为一整套有自己的律令和规则的僵硬、死板的制度,这的确是在这人世间以最严肃、认真的态度上演的一出闹剧,它是对愚昧的尊崇和膜拜。只不过这骑士荣誉的根本原则却是错误的。在处理其他无关重要的事情时(因为重要的事情由法庭处理),两个同样都是无所畏惧的人,其中的一个,亦即那更明智的一个,会作出让步,会同意保留各自的分歧。那些不认同骑士荣誉原则的普通大众,或者不计其数的来自各种阶层的人士的做法可以提供这方面的明证。他们让争执和磨擦顺其自然地得到解决。在这些人当中比起那或许只占整个社会人口的四分之一的信奉骑士荣誉的阶层,置人于死地的攻击要稀有一百倍,打架也是少有发生的事情。不过,会有人说:良好的礼仪和习惯归根到底建筑在骑士荣誉的原则以及由此而来的决斗之上,因为这些都是抵御人们不良举止和粗野行为的利器。但是,在雅典、哥林斯、罗马我们都的确见到良好甚至一流的社交氛围,以及优雅的礼貌和习惯,而这些并没有骑士荣誉这一鬼怪作其后盾。当然,在古时候,女人还没有在社交场合占据着显要的位置,就像我们现在的情形那样。现在出现的情形使人们的交谈多了一种轻浮、幼稚的东西,它赶走了有份量、严肃的话题;它确实在很大程度上造成我们的上流社会钟情于个人勇气更甚于其他品质。但是,个人勇气根本上只是次一级的,它只是行伍军人的优点。在个人勇气这一方面,甚至动物都超过我们,例如,人们说“像狮子一般勇敢”。与上述说法相反,骑士荣
誉的原则通常都为在大事上的不诚实和卑鄙的行为,在小事上的野蛮、欠缺考虑和不礼貌的做法提供了一个可靠的庇护所。因为人们会默默忍受许多的粗野行为,仅仅因为没有人愿意冒着失去生命的危险去批评别人。与我这里的说法相符的事实就是:在一个政治事务和财产金融事务中欠缺真正信誉的国家,我们可以看到决斗发展到了登峰造极、至为血腥的地步。至于这个国家的民间私下交往的情况,则可以向那些有个中经验的人询问。至于这个国家欠缺礼貌和社交修养,则是明白无误的事情。 

所有骑士荣誉的借口都经受不住检验。但如果有人说:正如一只狗遭到另一只狗的吠叫时,会以吠叫回应;但受到爱抚时,它就会表示亲热;同样,人的本性就是以敌意回应敌意,在遭受别人的蔑视或者憎恨的表示时,会内心难受和愤怒——那么,他的这番说词还会有点道理。因此,西塞罗说,“对于侮辱和恶待留下的疼痛,甚至谦逊和好心的人也难以承受。”无论在世界各地(除了信奉某些教派的人),人们都不会对别人的侮辱和拳头安之若素。尽管如此,人的天性只会驱使我们做出与我们所受到的冒犯相对应的报复,而不会比这走得更远;更加不会因为别人指责我们说谎、愚蠢和怯懦就致人于死地。古老的德国原则“耳光要以匕首偿还”表达的是令人反感的骑士观点。对遭受侮辱而做出报复或惩罚是因为愤怒的缘故,并不就像骑士荣誉所告诉我们的那样,它关乎我们的荣誉和道义。相反,确切无疑的是:指责我们的话语所造成的伤害程度是由这些话语击中目标的程度而定。这点可以从这样一个事实看得出来:只要别人说中了我们,那么,一个轻微的暗示所造成的伤害都更甚于一个严重的、但却没有事实根据的指责。所以,一个人一旦知道对自己的指责文不对题,那他就会
并且也应该自信地对此指责不屑一顾。但是,骑士荣誉的原则却要求我们承受我们其实并不应该承受的指责,并且针对这个并没有对自己造成任何伤害的侮辱采取血腥的报复。但如果一个人急急忙忙地压制每一句冒犯的话语,生怕这些话语被别人听见,那么,这个人肯定对自己的价值评价不高。因此,一个真正有自尊的人面对侮辱、诋毁会淡然处之;如果做不到漠然对待侮辱和诋毁,那么,机智和修养将帮助他顾全面子和掩藏起怒气。如果我们首先摆脱骑士原则的成见,不再误以为通过侮辱他人就可以夺取他人的荣誉或者挽回自己的荣誉;同时,如果人们不要随时报复,以泄心头之愤,亦即动用拳脚对付遭受到的各种各样的不公正、粗野蛮横的行为——因为这种回应马上就会使所有这类行为合法化,——如果真的是这样,那么,人们很快就会普遍接受这一观念:如果出现了恶语相向的情形,那占下风的一方就是胜利者。就像文圣佐·蒙蒂所说的:恶言秽语就像教堂的队列,永远返回它们的出发点。这样,人们就再不用像现在那样必须以牙还牙对付侮辱才可以保持自己的正确。这样,理解和思想才得以进入我们的交谈,而不是像现在那样:首先顾虑我们的说话是否得罪那些狭窄和愚昧的头脑。事实上,深刻理解力的存在本身就使狭窄、愚昧的人惊慌和难受,并由此引发一场在有思想、有头脑的人与皮囊之中充塞着肤浅、狭窄、愚昧的人之间展开的一赌运气的搏斗。这样,在人们的聚会中,精神的优势才会得到它所应得的优先权。但现在,这种优先权却给予了那些拥有一身蛮力和匹夫之勇的人,虽然这一事实并不为人所知。这样,出类拔萃的人起码就减少了一条逃避社交的理由。这种改变为真正的良好气氛和优秀的社交聚会扫清道路。毫无疑问,雅典、哥林斯和罗马曾经有过类似的聚会。谁愿意得到这方面的证据,那我就推荐他阅读色诺芬的《会饮篇》。 

不过,对骑士荣誉的最后辩护无疑是这样:“但,上帝啊,如果真的这样,每一个人不就都可以随便对他人动粗了吗?”——对此我能简略作答:这种情形发生在占社会人数百分之九十九并不奉行骑士荣誉的人群,但却不会有一个人因动粗而丧身。但在信守骑士原则的人群当中,一般来说一次动粗都会酿成致命。我还要深入地谈论这一问题。为了解释人类社会的一部分人所持有的这一根深蒂固的观点,即遭受别人的一巴掌是一件可怕至极的事,我曾努力试图找出存在于人类的动物性或理性的一些扎实、站得住脚,或者起码说得过去的理由,一些不是纯粹漂亮、花巧的言词,而是能被精练为清晰概念的理由。但我没有成功。动手打人一巴掌只是、并且永远是一个人对另一个人做出的肉体伤害行为,它表明出手打人的人身体更强壮或者动作更加敏捷,或者挨打的一方当时并没有留神等;除此之外,它并没有说明任何另外别的东西。对动手打人一巴掌这一行为的分析无法提供更多的东西。把遭受别人一记耳光视为一件最悲惨的事情的骑士,却挨过他的马匹比这一巴掌厉害十倍的踢打。但一蹶一拐的他,会强忍疼痛,一再安慰旁人那没有什么。这样,我想原因出在人手了。不过,我看到我们的骑士在战斗中受到了同样出自人手的剑刺、刀砍,他却向我们保证这些都是小菜一碟,不值一提。然后我们又听人家说,就算被人用马刀的平面拍打也远远没有挨别人的棍打那么严重。因此,不久前,军校的学生宁愿接受前者的惩罚而不接受后者。而时至今日,被马刀的刃面轻拍以获授骑士称号已成了一项至高的荣誉。现在我已完成对骑士荣誉的心理和道德理由的思考,剩下的只是这样一个结论:骑士荣誉的原则不过就是一个古老、根深蒂固的谬见,是说明人类的轻信特性的又一个例子。另外,一个众人皆知的事实可以证实我的观点:在中国,用竹杖抽打是司空见惯的一种惩罚公民的手段,甚至对各级官吏也是如此。这告诉我们,在中国,人性——那可是经过高度
文明教化的人性——并不赞同类似骑士荣誉的东西\footnote{在背上接受20或30竹杖子的抽打,可以说是中国人的家常便饭。这是中国人教育子女的方式,那并不是一件什么大不了的事情,被罚者亦以感谢的态度接受它们。——《教育和奇妙书信集》第二卷(1819)——原注。}。只要不带偏见地看一看人类的本性,就可以知道人与人打架是最自然不过的事情,这就犹如野兽间的厮咬和带角动物的竖角相撞;人不过就是会用鞭子打人的动物。因此,当我们偶尔听到一个人用嘴咬了另一个人,我们就会感到震惊,相比之下,动手动脚打架却是一件完全自然的事情。很明显,通过人们提高修养和发挥各自的自我克制,我们很乐意摒弃打斗的行为。但是,如果让一个国家或者只是一个阶层的人们相信:挨了别人一巴掌就是一件天大的不幸,那么,必然导致的结果就是致人于死地和相互谋杀。这是一件惨无人道的事情。在这世界上货真价实的祸害已经太多,人们不应该再增添那些虚幻不实的灾祸,因为它们会带来真正的祸害。但这正是那愚蠢和阴险的迷信正在做的事情。为此原因,我们抗议政府和立法机关为此鸣锣开道——他们热切地引入有关规定,禁止在民间和军队进行体罚。它们相信这样做会利益众生,但实情却恰恰相反。这种做法只会加剧那违反人性的、无可救药的愚昧。人们已为此付出了太多的牺牲。对于除了最严重的罪行以外的一般违法过失,人们首先想到的就是要给犯人一顿痛打。因此,这样的处罚是合乎自然的。谁要是不接受理智,那他就必须接受棍棒。

如果一个人既没有财产可供交付罚金,同时,剥夺他的自由也不会给人们带来益处——因为人们需要他的工作效劳——那么,对这个人施以适量的体罚,则是一件既明智又合乎自然的事情。我们对此并没有反对的理由,除了诸如“人的尊严和价值”一类的说辞。但支撑这些说词的并不是清晰的概念,而只是上文所讨论过的那种有害的谬见——它是问题的根源。关于这一点,可以从下面一个近乎可笑的例子得到证实:最近,在很多国家的军队里,鞭罚被睡板条床取代,后者和前者一样都给身体带来痛苦,但后一种处罚却不被认为有损受惩罚者的名誉和人格。 

人们如此鼓励这种盲目的观点,只会助骑士荣誉一臂之力,也因此助长了决斗的行为。与此同时,人们却又试图通过法律禁止决斗,或者似乎在这样做\footnote{政府似乎正在尽力消除决斗。虽然这是一件明显容易的事情,特别是在大学,但政府给人的印象是它并不想取得成功。其中的原因据我看来是这样的:国家无法以现金足够支付它的官员和民政官员;因此,政府就把该支付的另一半工资转化为颁发荣誉,而荣誉则通过头衔、制服、勋章等显现。为了更好维护这一理想的支付服务方式,政府就必须以各种可能的方式培养、加强人们的荣誉感,无论如何都要把荣誉感变成一件奢侈品。公民荣誉不足以满足政府的这一目的,因为大众都享有这种荣誉。因此,政府只能求助于骑士荣誉,并且以我所说的方式对它加以维护。在英国,因为政府对军队和民政服务的报偿比在欧洲大陆要高出许多,所以上面所说的补偿就不需要了。因此,在英国,尤其是在最近的20年,决斗几乎全然被废除了。决斗事件的发生只是极为个别稀有的事情。决斗只是作为一件蠢事受到人们的嘲笑。确实,“反对决斗团体”——这团队由许多的爵士、将军和司令组成——对这一结果贡献良多,莫洛赫神(古代腓尼基人所信奉的火神,以儿童作为献祭品。——译者)再也得不到祭品了。——原注。}。这样的结果就是:从最野蛮的中世纪流传下来的拳头即公理的认识残余,还游荡在19世纪。这实在是社会大众的耻辱。现在实在是到了把它羞辱一番然后弃如敝屣的时候了。当今已不允许人们斗狗了(至少在英国类似的娱乐遭到处罚),但人们却违反意愿地互相搏斗,致对方于死地。这都是荒谬的骑士原则和那些拥护骑士原则的思想偏执、狭窄的辩护人为其原则宣传、鸣锣开道的结果。他们强迫人们为了一些鸡毛蒜皮的事情,就要像角斗士一样地拼搏。因此,我向德语语言学者进言:duell这个词应由baiting这个字代替。这个字或许并不源自拉丁语duellam,而是出自西班牙语的duelo,意思是痛苦、艰难。决斗这种愚蠢的行为却以一本正经的方式进行,它不仅仅为人们提供了笑料。骑士荣誉的荒谬原则在一个国家里面另立一个独立王国,除了拳头即公理,其他一概不承认;它设置一个神圣的宗教裁判庭,对屈服在骑士荣誉的权威之下的各个阶层的人士施虐;每个人都可能因为一些微不足道的借口,受到他人的挑衅,从而被迫接受生存或者死亡的判决。所有这一切都令人愤慨。当然,它为恶棍们提供了庇护和藏匿之所——只要他们信奉骑士荣誉;他们可以威胁,甚至除掉那些高贵、卓越的人。那些人因为自身的高贵和卓越就会招惹这些恶棍的憎恨。时至今日,警察和法律已经使恶棍们不可能在大街上冲着我们喊道:“要钱还是要命?”同样,我们健康的理智也应该不再让恶棍破坏我们的平静,冲着我们喊叫:“要荣誉还是要性命?”上流阶层的人士应该解除负担,不要随时听任别人的随心所欲,为其野蛮、愚蠢或者恶毒付出代价,赔上自己的身体和生命。两个少不更事的青年人,相互间一旦出言不逊,就头脑发热,不惜付出鲜血、健康,或者生命。这是骇人听闻和令人羞愧的。很多时候,被侮辱者无法恢复受到损害的荣誉,因为他们与冒犯者的地位差别悬殊,或者因为冒犯者的某些特殊之处,这样,他们就在绝望中自己结束生命,落得个既悲哀又滑稽的下场。这个国中之国的暴虐和骑士荣誉这一谬见的威力由此可见一斑。如果事情的发展达至相互矛盾的顶点,那么,它的虚假和荒谬也就暴露出来了。这一例子就是一个明显的二律背反:官员被禁止参加决斗,但如果有人向他提出决斗而他又拒绝的话,他就会受到被解职的处罚。 

谈起这个话题,我就要老实不客气地说下去。只要我们不带成见地、清楚地审视这一问题,我们就可发现,我们手持与对手相同的武器,在光明正大的搏斗中把对方杀死,抑或从背后暗袭得手——这两者之间之所以存在重要的差别,并且人们如此高度地重视这一差别,原因其实全在于这一事实:在这国中之国,正如我已经说过的,人们承认强者的权利亦即拳头即公理,把拳头即公理尊奉为上帝的裁决,并把它作为骑士荣誉的规则基础。在公平的搏斗中杀死我们的敌人,除了证明我们身强力壮,或者更具击斗的技巧以外,别无其他。进行公开决斗以证明杀死对手就是占理的,这也就假设了这样的前提:强力就是真的公理了。但事实上,如果我的对手并不懂得自我防卫,那就只为我提供了杀死他的可能,而绝没有提供杀死他的正当理由。恰恰相反,我杀死对手的道义上的理由只能取决于我要杀死他的动机原因。假设我杀死对手在道德上是足够合理的,那么,让杀死他的这件事取决于我是否在射术或者击剑上优胜于对方,是完全没有理由的。相反,我到底采用何种方式夺取他的生命,从后面抑或从前面袭击他,结果都一样。如果要对某人实施卑鄙的谋杀,那诡计就应该派上用场。从道德的角度看,强力即公理并不比诡计即公理更令人信服。就我们现在所说的情形而言,强力即公理和诡计即公理并没有两样。需
要注意的是,在决斗中强力和诡计都在发挥作用,因为击剑中的花招都是阴招。如果我认为杀死一个人在道义上是合理的,那么,杀死他这样的事情由他和我到底是谁更精于射击、击剑来作决定,则是愚蠢的;因为那样的话,对手不仅会反过来伤害我,甚至会夺去我的性命。报复别人的侮辱不应该采用决斗的方式,而应该运用暗杀的手段——这是卢梭的看法。他在《爱弥儿》第四部的相当神秘的第二十一条注释里面小心翼翼地暗示了这一看法。但他深受骑士荣誉的影响,竟然认为如果被人指责说谎,那自己就有了正当的理由暗杀这个人。但卢梭应该知道:每一个人都无数次说过谎,都配受到这一责备,卢梭自己本人就更是如此。一个人只要是光明正大地、以相同的武器和对手较量,那他杀死他的对手就是正当合法的——这一谬见明显把强力当成真的公理了,而决斗就被认为是上帝的判决了。相比之下,一个怒不可遏的意大利人见到自己的仇人,二话不说就会扑上去用匕首袭击敌人。这一行为做法起码是连贯一致、合乎自然的;他更聪明,但却不会比参加决斗者更卑劣。但有人会说:在决斗中,当我杀死我的对手时,他其实也在试图置我于死地,这足以为我开脱责任。但对此的驳斥是,当我向他发出挑战时,我也就已经把他置于不得不正当护卫自己的境地。这种故意将对方置于如此境地的做法,事实上就是
决斗者在为谋杀对方寻找一个说得过去的借口而已。如果双方同意把生命押在决斗上面,那用自愿的行为属于咎由自取这一原则作为开脱则更说得过去。对此我们可以说,受损害的一方并不就是自愿的,因为杀人的刽子手是那暴虐的骑士荣誉及其荒谬的规则。它把两个决斗者,或者至少其中一个拉到了这血腥的私设刑庭的前面。

我对骑士荣誉的讨论扯得太长了,但我这样做实在是用心良苦,因为哲学是这世界上唯一能够对付道德和智力范畴的庞然大物的大力神。两样主要的东西把新时代的社会与古老社会区别开来,并把前者比了下去,因为这两样东西使新时代社会的人带上某种阴沉、严肃和不祥的神气。在古老的时候,人们可没有这一弊病,那个时期就犹如生命中的早晨,快乐和不受拘束。这两样东西就是骑士荣誉和性病,这“高贵的一对”(贺拉斯语)。它们携手毒害了生活中的“辩论和爱情”。性病发挥的影响要比乍看上去的深远,因为它的影响并不纯粹是生理上的,而且还是道德上的。既然在丘比特的箭袋里也有带毒的箭,那么,男女两性之间的关系也就掺进了某种陌生的、敌视的甚至魔鬼般的成分。这样,阴暗、可怕的不信任就由此进入两性间的关系。构成所有人类社会的基础如今发生了这种变化,这都会或多或少间接影响到其他的社会关系。但深入探讨这一问题则会偏离我们的题目。与性病的影响相类似的是骑土荣誉的影响,虽然其性质有所不同。社会受到它的影响而变得僵硬、紧张和严肃,因为人们每说一句话都必须煞费思量。但这些还不是事情的全部!骑士荣誉原则是公众供奉的牛首人身的弥诺佗,每年供奉给他的祭品是许多出自名门的高贵男儿。这情形并不像过去那样,只发生在欧洲某一国家,它已经遍及整个
欧洲。因此,现在是到了勇敢剿灭这只妖魔的时候了,就像我在这里所做的那样。就让这两只怪物在19世纪的新时代寿终正寝吧!医生最终会通过预防药物成功医治性病,对此我们不会放弃希望;但要消除骑士荣誉这只妖魔却是哲学家的任务,他们必须纠正人们的观念,因为,政府所运用的立法手段至今为止已告失败。并且,也只有通过哲学才能从根本上打击这种祸害。如果政府真心实意地展开工作,杜绝决斗这一祸害,而收效甚微的原因的确只是因为政府无能为力,那么,我将向政府建议一条法律,我保证它会获得成功。我们用不着采用血腥的手段,也不必求助于断头台、绞刑架和终身囚监等方法。相反,它简单易行,是一种“顺势疗法”。如果有人向别人发出决斗的挑战,或者接受别人的挑战,那就让他在光天化日之下,在士兵长之前,像中国人那样接受执罚者的12杖的体罚。为决斗者传递挑战书的人和公证人则每人承受6杖的处罚,对于决斗所造成的后果,则循惯常的刑事诉讼法追究责任。或许,一个脑子里充满骑士思想的人会反驳说:经此体罚,许多“荣誉之士”就会开枪了断自己。对此说法,我的回答是:这样的一个傻瓜杀死自己总比杀死别人要好。根本上我就知道得很清楚,政府并没有真心实意地杜绝决斗。民政官员,尤其是一般的官员(最高职位的官员除外)的收入远远低于他们的服务所应获得的数目。因此,他们的另一半收入就以荣誉支付了。荣誉首先通过头衔和勋章来表示,其次,在更广泛的意义上,以社会阶层的荣誉为代表。决斗对于社会阶层所代表的荣誉是拉曳马车的一匹得力副马。因此,在大学里人们就已经接受关于荣誉的初步的训练了。所以,决斗的受害者其实是以自己的鲜血弥补了工资收入的不足。 

为使我的讨论更加完整,我需要约略提及民族的荣誉。它涉及整个民族——那构成了人类社会的一部分。在民族荣誉的问题上,力量是唯一的裁决者,舍此之外,别无其他。因此,民族的每一个成员都必须自觉保卫他的民族的权利。所以,民族荣誉并不只是需要别人认定:这一民族是值得信赖的。除此之外,它还要人们知道:这个民族是令人生畏的。因此,民族荣誉不容许听任外族入侵犯本民族的权利而置之不理。这样,民族荣誉就结合了公民荣誉和骑士荣誉。 

我在人们给予的表象——亦即人们在他人眼中的样子——一部分里最后提到过名声。在此,我们必须继续对它作一番考察。名声和名誉是一对孪生兄弟,但就像第奥斯科所生的孪生子一样:一个(波鲁斯)长生不老,另一个(卡斯图)则终究要死亡。名誉是可朽的,名声就是名誉的不朽的兄弟。当然,我这里所说的名声指的是级别最高、货真价实的那一种;因为太多的名声只是过眼云烟而已。名誉只包含人们在同等的处境下必须具备的素质,每一个人都应公开视这些素质为己所有。但名声涉及的素质则是我们不可能要求人们一定具备的。名誉尾随着别人对于我们的了解,而不会超越此界限;但名声却与此相反,它走在别人对我们的了解之前:并且,它把名誉也带到了名声抵达之处。每一个人都可以拥有名誉,但只有少数例外的人才能获得名声,因为名声的获得只能通过做出行动业绩,或者创作思想性的作品。这是获得名声的两条途径。要建功立业,就必须具备一颗伟大的心,但创作巨著则需要拥有一个非凡的头脑。这两条成名之路都有各自的优点和缺陷,但两者的主要差别在于事功会消逝,但作品却可以永存。最高贵的事功也只具有暂时性的影响。但天才的作品却能传之久远,给人以教益和愉悦。事功留给人们的是记忆,并且,除非历史把事功业绩记录下来,像化石一样地传给后世,否则,这一记忆就会永远不断地减弱、变型,最终变得模糊以致湮没。相比之下,作品的自身就是不朽的,文学著作尤其能够世代相传。关于亚历山大大帝,现在仅存的是他的名字和对他的记忆。但柏拉图、亚里士多德、荷马、贺拉斯却仍然活生生地存在着,仍然在直接地产生着影响。《吠陀》及其《奥义书》仍然存在。但对于过去各个时代所产生的行动业绩我们却已经一无所知了\footnote{因此,人们以为将作品冠以行动业绩之名就可以使作品享誉——这是人们今天的时尚做法。其实,这一恭维方式糟糕至极,因为作品从本质上而言,就是高于行动业绩的。行动业绩永远都只是服务于动机的行动,因此,它具有局限性,并且匆匆即逝;它属于这个世界的普遍、原始的成分,亦即隶属于意欲。但一件伟大或者优美的作品却是永存的,因为它包含广泛、普遍的意义,它发自智慧,纯粹,无瑕,就犹如从这一意欲世界升华起来的一缕芬芳气息。由行动业绩带来的名声自有其优势。这种名声一般都伴随着强烈的轰动。很多时候,这种轰动、雷鸣足以传遍整个欧洲。但通过作品获致的名声,其到来却是逐步和缓慢的。在开始的时候,它的声音是微弱的,然后才逐渐响亮起来。这种名声通常必须历经一个世纪之久,才能达到它最显赫的时候。不过,因为作品维持长久,所以,这种名声有时维持上千年之久。但行动业绩所带来的名声,在经过最初的震耳欲聋的鸣响以后,声音就逐渐减弱,少为人知,逝去不返了。到最后,只能像幽灵一般地存在于历史之中。——原注。}。行动业绩的另一个不便之处就是它们有赖于机会。因为机会首先为行动业绩的发生提供了可能;这样,通过行动业绩获取的名声就并不由这行动业绩的自身价值所决定,而是根据当时的情势而定。因为正是当时的情势使行动业绩具有了重要性和得到了荣耀。此外,如果行动业绩纯粹属于个人行为,例如在战争中,那么,它就全凭为数不多的目击证人的描述;但是,目击证人也不总是存在的,而且,他们也并不是公正无私、不带偏见的。不过,行动业绩也有其优势,那就是作为实际事务,它们处于普通大众评判能力的范围之内。因此,只要掌握了有关行动业绩的精确资料,人们马上就会给予这些行动业绩公正的承认——除非人们只是在以后才正确地认识和理解做出行动业绩背后的动机,因为只有对一桩行动业绩的动机有了认识以后才会理解这一行动业绩。对于创作作品,则是相反的情形。作品的形成并不依赖机会,它们只是依靠作品的创
作者本人。只要作品还存在,它们就以自身原来的样子而存在。不过,评判作品存在一定的困难。作品的级别越高,评判这些作品的难度就越大。我们通常都缺乏具有才气、不带偏见和诚实正直的评判员。作品的名声不会因为一个评判或者一桩事件而一锤定音。作品有一个上诉的过程。就像我已经说过的,行动业绩通过记忆传达给后世,并且,其传达方式由这些行动业绩发生时候的一代人提供。但作品除非缺失了某些部分,否则,就以自身原来的样子留传下来。这样,我们就不会歪曲作品的面目。并且,作品在创作和面世时所遭遇的当时情势环境的不利影响,会在以后的时间消失。另外,时间还带来了为数不多的真正具备能力的评判员。他们本身就是非凡的人物,现在他们评判的是比自己更加非凡和出色的作品。他们各自给予有相当分量的意见。当然,有时候历经数世以后,才会产生完全公正的评判结果,而这一定论是不会被将来推翻的。由作品奠定的名声是牢固和势所必然的。不过,作品的作者能否亲眼目睹自己的作品获得承认,却取决于外在情势和一定的运气。作品越高贵、越有深度,这种情况就越少发生。塞尼加曾经很好地谈论过这一点。他说,名声跟随成就如影随形,但当然,像影子那样忽而在前,忽而在后。他说清楚这点以后,又加上了这么一句:“虽然嫉妒让你的同时代人沉默,但以后总会有人不带恶意,也不带恭维地作出判断。”顺便说上一句,从这里我们可以看到在塞尼加的时代,无赖们就已经施展这种压制成就的艺术了,那就是:对别人的成就保持恶意的沉默,采取视而不见的态度。他们以这种方式不让公众看到优秀的事物,这更有利于那些低微、拙劣的东西。他们熟练运用这一艺术丝毫不亚于我们的当代人。嫉妒使他们和我们这个时代的无赖们都闭上了嘴巴。一般来说,名声到来越迟,维持的时间就越长久,因为任何优秀的东西都只能慢慢地成熟。流芳后世的名声就好比一株慢慢成长起来的橡树。那得来全不费工夫、但却只是昙花一现的名声,只是寿命不过一年的快速长成的植物;而虚假的名声则是迅速茁壮起来,但却很快就被连根拔掉的杂草。这都是由这一事实决定的:一个人越属于他的后世,亦即属于整个人类大众,那他就越不为自己的时代所了解,因为他的贡献对象不仅是他的时代,他为之奉献的是整个人类。因此,他创作的作品并不会沾上局限于自己时代的色彩。因此原因,很容易就会出现这种情况:他默默无闻地度过了他的时代。而那些只为短促一生中的事务效劳、只服务于刹那瞬间的人——他们因此属于他们的时代,并且与这个时代同生共死——反而会得到他们同时代的人的赏识。所以,艺术史和文学史告诉我们:人类精神思想的最高级的产物一般都得不到人们的欢迎,这种情况一直维持到优秀的思想者的出现——他们感受到了这些作品发出的呼唤,并使这些作品获得了威望。凭藉如此得来的权威,这些作品也就可以继续保有其威望了。之所以出现这种情形,根本原因就是每一个人都只能理解和欣赏与自己的本性相呼应的东西。一个呆板的人只能理解呆板事物,一个庸俗的人只会欣赏平庸和俗套,头脑混乱者喜欢模糊不清的东西,没有思想的人则和胡言昏话气味相投。与读者本人同气相通的作品最能够获得这个读者的欢心。因此,古老的、寓言式的人物伊壁查姆斯唱道(我的译文): 

 
\begin{quotation}
我说出我自己的看法,这毫不奇怪;

而他们沾沾自喜,自以为 

他们才是值得称道的。狗对于狗来说, 

当然才是漂亮的生物。牛对牛也是这样, 

猪对于猪、驴子对驴子,莫不如此。 
\end{quotation}


就算是最强有力的手臂甩出一件很轻的物体,也无法给予这轻物足够的力量让它飞得很远,并且有力地击中目标。这轻物很快就会坠落地面,因为这轻物本身没有物质性的实体以接收外力。美妙和伟大的思想、天才创作的巨作也会遭遇同样的情形——如果接受这些思想的都只是弱小、荒诞的头脑。各个时代的智者们都曾为此齐声哀叹。例如,耶稣说:“向一个愚人说故事,就像跟一个睡着觉的人说话一样。故事讲完了,他会问,你说的是什么?”哈姆莱特说:“机智妙语在傻瓜的耳朵里面睡觉。”歌德说过: 

 
\begin{quotation}
最妙的话语,被笨蛋听了, 

也会招来讽刺。 
\end{quotation}


又 

\begin{quotation}
你的话没有任何结果, 

众人都呆滞无言, 

保持良好的心情吧! 

石头扔进沼泽地?

是不会弄出涟漪的。 
\end{quotation}

 

利希腾贝格说道,“当一个脑袋和一本书互相碰撞,而只发出空洞的闷响,这空响难道就总出自书本?”他还说,“书本身就是一面镜子,一只猴子看镜子的时候,里面不会出现福音圣徒。”的确,吉拉特神父对此的优美和感人的哀怨值得让人回味: 

 
\begin{quotation}
最好的礼物通常 

最不被人赞叹; 

世上的大部分人, 

把最坏的视为最好。

这一糟糕的状况司空见惯, 

但人们如何避免这种不幸? 

我怀疑能否从我们的世界根除这一不幸, 

世上只有唯一的补救之法,但它却无比困难: 

愚人必须获得智慧——但这是他们永远无法做到的, 

他们也不会懂得事物的价值。 

作出判断的只是他们的眼睛,而不是脑袋, 

他们赞扬微不足道的东西, 

只是因为他们从来不曾懂得什么才是好的。 
\end{quotation}


由于人们思想水平的低下,所以,正如歌德所说的,优秀人物很少被人发现,他们能够获得人们的承认和赏识就更是稀奇的事情。人们除了智力的不足,还有一种道德上的劣性:那就是嫉妒。一旦一个人获得了名声,那名声就会使他处于高于众人的位置,而别人也就因此被相对贬低了。所以,每一个作出非凡成绩和贡献的人所得到的名声是以那些并不曾得到名声的人为代价的。 

 
\begin{quotation}
我们给予别人荣誉的同时, 

也就降低了我们自己。 

——歌德 
\end{quotation}


由此我们可以明白,为何优秀出色的东西甫一露面,不论它们属何种类,就会受到数不胜数的平庸之辈的攻击。他们联合起来,誓要阻止这些东西的出现;甚至尽其所能,必欲去之而后快。这群大众采用的暗语就是“打倒成就和贡献”。甚至那些做出了成绩并以此得到了名声的人,也不愿看到其他人享有新的名声,因为其他名声所发出的光彩会令他们失色。因此,歌德写道: 

 
\begin{quotation}
在得到生命之前, 

倘若我踌躇一番, 

我就不会活在这世上了。 

正如你们看到了, 

那些趾高气扬的人,为了炫耀自己, 

就要忽视我的存在。 
\end{quotation}


一般来说,名誉会得到人们公正的评判,它也不会受到嫉妒的攻击,事实上名誉都是预先给予每一个人的;但获得名声只能经过与嫉妒的一番恶斗,并且,月桂花环是由那些绝非公正的评判员所组成的裁判庭颁发授予的。人们能够而且愿意与别人一道享有名誉,但获取了名声的人却会贬低名声,或者阻挠别人得到它。另外,通过创作作品而获取名声的难度与这一作品的读者群的数目成反比,个中的理由显而易见。创作旨在给人以教益的作品比起写作供人们娱乐消遣的作品更难获取名声。撰写哲学著作以获取名声是最困难的,因为这些著作给
人们的教益并不确定;另外,它们也没有物质上的用处。所以,哲学著作面向的读者群全由从事哲学的同行所组成。从上述的困难可以想见,那些写作配享声誉的作品的作者,假如不是出于对自己事业的热爱,并且在写作的时候能够自得其乐,而是受着要获取名声的鼓动去写作,那么,人类就不会有,或者只会有很少不朽的著作。的确,要创作出优秀的著作,并且避免写出低劣的作品,创作者就必须抵制和鄙视大众及其代言人的评判。据此,这一说法相当正确——奥索里亚斯尤其强调这一说法——名声总是逃离追逐它的人,但却会尾随对它毫不在意的人。这是因为前者只投合自己同时代人的口味,但后者却抵制这种口味。 

因此,获取名声是困难的。但保存名声却非常容易。在这一方面,名声和名誉恰成对照。名誉是预支给每一个人的,每一个人只需小心呵护它就是了。但问题是,一个人只要做出某一不端的行为,他的名誉就一去不复返了。相比之下,名声不会真正失去,因为一个人赖以取得名声的业绩或作品总是摆在那里,尽管它们的创造者不再有新的创作,但名声仍然伴随着他。如果名声真的减弱、消失,变成了明日黄花,那么,这一名声就不是真的,也就是说,这名声不是实至名归的,它只是由于暂时获得了过高的评价所致;要么,它干脆就类似黑格尔所取得的那种名声——利希腾贝格对此有过描述:“它由那些好友集团齐声宣扬,然后得到了空洞的脑袋的回应……当将来有朝一日,后人面对那些花花绿绿的言语大厦,还有逝去的时髦所留下的漂亮空壳,以及死掉了的概念所占据的框架子,当他们敲门时竟发现一切全是空架子,里面甚至没有点滴的思想能够有信心地喊出:请进来吧——这将沦为怎样的笑柄啊!”(《杂作》4,15页) 

名声建立在一个人与众不同的地方。因此,名声本质上就是相对比较而言的,它也只具备相对的价值。一旦其他人和享有名声者都是同一个样子,那名声也就不复存在了。只有那些在任何情况下都能保有其价值——在这里,亦即自身直接拥有的东西——才具备绝对的价值。因而,伟大的心和伟大的头脑所具备的价值和幸福全在于它们的自身。具有价值的不是名声,而是藉以获得名声的东西——它才是实在的,而以此获得的名声只是一种偶然意外而已。的确,名声只是某种的外在显示,名人以此证实了自己对自己所抱有的高度的评价并没有
错。因此,人们可以说:正如光本身是看不见的,除非它经过物体的折射,同样,一个人所具有的卓越之处只是通过获得名声才变得无可争议。不过,名声这种外部显示可不是万无一失的,因为盛名之下,其实可能难副。另外,做出了非凡贡献的人却有可能欠缺名声。所以莱辛\footnote{我们最大的乐趣在于得到别人的赞叹;但尽管羡慕者有充分的理由羡慕别人,但他们并不愿意表露自己的羡慕之情。所以,最幸福的人就是能够做到真正赞叹自己的人,不管他以何种方式做到这一点。只要别人不让他对自己产生怀疑就行了。}的话说得很聪明:“一些人声名显赫,另外的一些人却理应声名显赫。”另外,如果一个人是否具备价值只能取决于这个人在别人的眼中所呈现的样子,那这样的生存将是悲惨的。如果一个英雄或者天才所具有的价值真的只在于他所拥有的名声,亦即在于他人对他的首
肯,那么,他的一生就确实够悲惨的了。但真实的情形却恰恰不是这样。每个人都根据其自身本性而生存,因此,他首先是以自身的样子为了自己而活。对于一个人来说,他的自身本性,不管其存在方式为何,才是最重要的东西。如果这个人的自身本性欠缺价值,那他这个人也就欠缺价值。相比之下,他在别人头脑中的形象却是次要和枝节的东西,它受制于偶然,对他本人也只能施加间接的影响。除此之外,大众的头脑是可怜、凄凉的舞台,真正的幸福不可能在这里安家落户,只有虚幻不实的幸福才会在这里栖身。在名声的殿堂里,我们可以看到多么混杂的各式人等啊:统帅、大臣、舞伎、歌手、伶人、富豪、庸医、犹太人、杂耍艺人等等。是的,在这里,所有这些人的过人之处比起不一般的精神思想素质——尤其是高级的一类——更能受到人们真诚的赏识和由衷的敬意。对于杰出的精神思想素质,绝大部分人只是在口头上表示敬意而已。从幸福学的角度看,名声只是喂养我们的骄傲和虚荣心的异常稀罕、昂贵的食物;除此之外,它就什么都不是了。但大多数人都有过度的骄傲和虚荣,虽然他们会把它掩饰起来。或许那一类不管怎么说都理应获取名声的人——他们的骄傲和虚荣才是最强烈的;在这些人的不确定的意识里,他们认为自己的价值优于常人。在获得机会去证实自己的突出价值并且获取承认之前,他们必须在漫长的时间里、在不确切之中等待。他们觉得遭受了某种不为人知的、不公正的对待。不过,一般来说,正如我在这一章开始的时候已经说过的,人们重视别人对自己的看法的程度,是完全失去比例和不合理智的。所以,霍布斯的言词虽然表达得相当强烈,但却或许是正确的:“我们心情愉快就在于有可供与我们比较并使我们可以看重自己的人。”由此可以明白为什么人们如此看重名声,并且为了最终得到名声而付出种种牺牲: 


\begin{quotation}
名声(这是高贵的心灵最后的弱点) 

促使清晰的头脑鄙视欢愉, 

过着辛劳艰苦的日子。 

——弥尔顿(卢西达斯),70 
\end{quotation}
 

另外, 

 
\begin{quotation}
高傲的名声殿堂闪耀在 

陡峭的山上,

人生的智慧要爬上去是多么的艰难! 

——贝蒂(吟游诗人) 
\end{quotation}
 

最后,我们也可以看出,最虚荣的国家总把荣耀挂在嘴上,并毫不迟疑地把它视为激励人们做出非凡的实事和创作出伟大著作的主要原动力。但无可争辩的事实却是:名声只是一种次要之物,它只不过是成绩贡献的映象、表征、回音;并且,能够获取赞叹之物比赞叹更有价值。所以,让人们得到幸福的并不是名声,而是藉以获得名声的东西;因而,它在于成绩、贡献本身,或者,更准确地说,让人得到幸福的是产生出这些成绩和贡献的思想和能力,不管这两者的性质属于道德方面抑或智力方面。因为每个人为着自己的缘故都有必要发挥自己最出色的素质。他反映在别人头脑中的样子,以及别人对他的评价,其重要性都是次一级的。因此,配享名声而又不曾获得名声的人,其实拥有了那更加重要的东西;他所缺乏的尽可以用他的实际拥有作为安慰和弥补。我们羡慕一个伟人,并不是因为这个人被那些缺乏判断力、经常受到迷惑的大众视为伟人,而是因为这个人确实就是一个伟人。他的最大幸福并不在于后世的人会知道他,而在于在他那里我们看到了那些耐人琢磨、值得人们永久保存的思想。他的幸福是被自己所掌握的。但名声却不在“自己的掌握之中”。在另一方面,假如他人的赞叹才是最重要的,那么,引起赞叹的东西的重要性就配不上赞叹本身了。虚假的、名不副实的名声就属于这种情形。获得这种虚假名声的人享受名声带来的好处,但却并不真正具备名声所代表的东西。但虚假的名声也有变了味的时候。尽管为了自身的利益,这些人自己欺骗自己,但处于自己并不适应的高度,他们会感到阵阵的晕眩;或者,他们会觉得自己不过就是一个赝品而已。他们害怕最终被人剥落面具和遭受罪有应得的羞辱,尤其在有识之士的额头,他们就已经读到了将来后世的判决。这些人就好比伪造遗嘱骗取了财产的人。最真实的名声,亦即流传身后的名声,并不会被这名声的主人所知晓,但人们仍然会认为他是一个幸运的人。他的幸运就在于他具有藉以获取名声的非凡素质,同时,也在于他能有机会发展和发挥了这些素质,并能以适合自己的方式行事,从事他满怀喜悦地投身其中的事情,因为,只有这样产生出来的作品才能获取后世的名声。他的幸运还在于他具有伟大的情感或者丰富的精神世界,这些在他的作品中留下了印记,并获得了后世人们的赞叹。还有就是他的思想智慧。思考、琢磨他的思想智慧,将是在以后无尽的将来那些具有高贵思想的人们所乐于从事的工作。流芳后世的名声的价值在于这一名声的实至名归,这才是这种名声的唯一报酬。至于获取身后名声的作品是否也能博得作者同时代人的赞赏则视乎环境、运气,但这一点并不很重要。按照一般的规律,常人缺乏独立判断,尤其缺乏欣赏高级别和高难度的成就的能力,所以,人们就总是听从他人的权威。高级别的名声纯粹建立在称赞者的诚信之上,百分之九十九都是这样的情形。因此,对于那些深思的人来说,同时代喧哗的赞美声价值很低,因为他们听到的不过是为数不多几个声音在引起回响罢了。而这为数不多的几个声音也不过是一时的产物。如果一个小提琴手知道:他的听众除了一两个以外,都是由聋子组成,这些聋子为了互相掩藏自己的缺陷,每当看到那例外的一两个人双手有所动作就跟着热烈的鼓掌回应,那么,这个小提琴手还会为他的听众所给予的满堂掌声而高兴吗?甚至当他终于知道,那带头鼓掌的人经常被人行贿收买,为那可怜的小提琴演奏者制造出最响亮的喝彩声!由此看出,一个时代的名声何以极少转化为身后的名声。这就是为什么达兰贝尔在其对文学殿堂的优美描写中指出:“文学殿堂里住满了死去的人,他们在生前并不曾住在里面;这殿堂里面也有为数不多的几位生存者,但一旦他们死去,他们就几乎全部被抛出殿堂。”在这里顺便说上一句,在一个人的生前就为他竖立纪念碑——这就等于说:我们不放心后世去评价他的价值。但如果一个人真能在生前就享受到延绵后世的名声,那这种事情就绝少发生在他达致高龄之前。或许,这一规律的例外情形更多发生在艺术家和文学家的身上,但却甚少发生在哲学家身上。那些通过著作成名的人的肖像就为我的这一说法提供了例证,因为那些肖像大多是成名以后才准备的:这些肖像一般都表现着作者年老的模样,有着花白的头发,尤其是哲学家。但从幸福论的角度看,这又是绝对理所当然的事情。对于我们凡夫俗子来说,名声和青春加在一起简直太奢侈了。我们生活这样的贫乏,我们应该珍惜生活的赐予,把它们分开享用。在青春期,我们已经拥有足够的宝贵财富,并能以此得到快乐。但到了老年,当所有的快乐和欢娱犹如冬天的树木一样凋谢以后,名声之树就犹如冬青一般适时地抽芽长叶了。我们也可以把名声比作冬梨——它们在夏天生长,但在冬天供人享用。到了老年,我们没有比这更加美好的安慰了:我们把全部的青春力量都倾注到著作里面,这些著作并不会随着我们一起老去。 

现在,让我们更仔细地考察一下在一些与我们密切相关的学科获取名声的途径,我们也就可以得出下面的规律。要在这些学科表现出聪明才智——这方面的名声是其标识——就必须对这些学科的资料进行新的组合。这些资料内容性质各自差异很大,但这些资料越广为人知和越能被人接触,那么,通过整理和组合这些资料而获取的名声也就越大。例如,如果这些资料涉及的是数字或曲线,包含的是某些物理学、动物学、植物学或者解剖学方面的事实;又或者,如果这些研究资料是古代作家的散佚断篇,或者是些缺字短章的碑文、铭刻;又或
者,这些材料涉及历史的某一个模糊不清的时期,——那么,对这些资料进行亍番正确无误的整理和组合以后赢得的名声,则只流行于对这些资料有所认识的人群,而不会越出这个范围。因此,这类名声只在少数的、通常过着隐居生活的人之间传播。这些人对于别人享有他们这一专业行当的名声都心存嫉妒。但是,如果研究的资料众人皆知,例如,它们涉及人的理解力、人的感情的基本和普遍的特性,或者研究人们举目可见其发挥作用的各种自然力、人们耳熟能详的大自然的进程,那么,对这些资料进行重要的、令人耳目一新的组合,并以此扩大人们对这些事物的了解——通过这样的工作而获得的名声就会随着时间传遍整个文明世界。因为每个人都接触得到这些研究素材。所以,在大多数的情况下,每个人都可以对它们进行组合。因而,名声的大小总是与我们所要克服的困难的大小互相吻合。既然研究的资料广为人知,那么,采用崭新的、但却是正确的方式对它们进行组合就变得越加困难,因为太多的人已经在这一方面花费过脑筋,各种新组合的可能也已穷尽。与此相比,对于那些只能通过艰辛、困难的方式才能掌握的、不为一般大众所接触的研究资料,我们总可以找到这些素材的新的组合。所以,假如一个人有着清晰的理解力和健康的判断力,再加上一定的智力优势,那么,如果他从事上述这一类资料的研究,他就很有可能终于幸运地找到这些资料新的和正确的组合。不过,以这种方式获取的名声的流传范围或多或少是和人们对这类资料的了解、熟悉程度相一致的。解决这一类学科的难题要求人们进行大量的研究工作——仅仅了解和掌握这些资料就必须这样做了。但假如我们探究的是一类能够带给我们最显赫和最深远名声的资料,那么,这类资料素材的获得简直就是不费吹灰之力。但是,解答这类难题所需要的苦干越少,它对研究者的才能的要求就越高,甚至只有天才才足以胜任这一类工作。在创造的价值和受到人们的尊敬方面而言,苦干根本不能与思想的天才相提并论。 

由此可知,那些感觉自己具有良好的理解力和正确的判断力,但又不相信自己真的具备至高的思想禀赋的人,不应该惧怕从事繁琐的考究工夫和累人的工作,因为只有凭藉这些劳动,他们才能在广泛接触这些资料素材的众人当中脱颖而出,才能深入只有勤勉的博学者才有机会涉足的偏僻领域。在这一领域,竞争者的数目大为减少,具有稍为突出头脑的人都会很快找到机会对所研究的资料进行一番新的和正确的组合。这种人发现的功劳甚至就建立在他克服了困难而获得了这些资料上面。但是,大众只能遥远地听闻他由此获得的喝彩声——这些喝彩声来自他的研究学问的同行,因为只有这些人才懂得这一门专业。如果沿着我这里所说的路子一直走到底,最终就会由于发掘新的资料变得极其困难,研究者用不着组合资料了,他们只需找到资料就足以建立名声。这犹如一个探险家抵达一处偏僻、不见人烟的地方:他的所见而不是他的所想就会使他成名。这条成名途径还有一个很大的优势:传达自己的所见较之于传达自己的所想难度更小;对于理解他人的所见也较理解他人的所想更加容易。所以,讲述见闻的作品比传达思想的著作能够拥有更多的读者,因为,正如阿斯姆斯所说的: 

 
\begin{quotation}
一个人去旅行, 

就能讲故事。 
\end{quotation}


不过,与此相吻合的事情却是:私下认识和了解这一类著名人物以后,我们常会想起贺拉斯所说过的话: 

 
\begin{quotation}
到海外旅行的人只是变换了气候而已,他们并不曾改变思想意识。 
\end{quotation}
 

至于那些头脑天赋极强的人,因为他们应该去解答重大的难题,亦即那些涉及这个世界的普遍和总体方面、因此也是最困难的问题。所以他们应该尽可能地扩展视野,同时兼顾多个方向,以避免朝着一个方向走得太远而迷失在某一专门的、少为人知的领域里面。也就是说,他不要太过纠缠于某一学科之中的某一专门领域,更不用说去钻那些琐碎的牛角尖了。他不需要为了抛开那为数众多的竞争者而投身于偏僻的学科。每个人都能看得见的事物其实都可以成为他研究的素材。他可以对这些素材进行全新的、正确的和真实的组合。这样,他作出的贡献就能为所有熟悉那些资料素材的人欣赏,也就是说,获得人类的大多数的欣赏。文学家和哲学家获得的名声与物理学家、化学家、解剖学家、矿物学家、动物学家、语言学家、历史学家所得到的名声之间存在巨大的差别,道理全在这里。 




% 编者:万泽
\end{document}


